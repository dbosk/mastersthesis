\chapter{De hela talen}
\label{ch:Heltalen}
\noindent
\lettrine{M}{�ngden av} hela tal \(\Z=\{0, -1, 1, -2, 2, \ldots\}\).

...

\begin{definition}
	\index{invers}
	Givet en m�ngd med en definierad operation \(\diamond\) och ett
	identitetselement \(e\).
	Ett element \(x\) kallas f�r \emph{inversen till \(y\)} om
	\(y\diamond x = x\diamond y = e\).
\end{definition}

\section{Algebraiska egenskaper hos de hela talen}
\noindent
...

\begin{definition}[Algebraiska egenskaper hos de hela talen]
	\nocite{Bartle2000itr,Grillet2007aa}
	\label{def:HeltalenEgenskaper}
	P� m�ngden \(\Z\) definieras tv� bin�ra operationer, addition (\(+\)) och
	multiplikation (\(\cdot\)).
	F�r addition g�ller f�ljande:
	\begin{description}
		\item[Kommutativitet] \(a+b=b+a\) f�r alla \(a,b\in\Z\).
		\item[Associtivitet] \((a+b)+c=a+(b+c)\) f�r alla \(a,b,c\in\Z\).
		\item[Additivt identitetselement] Det finns ett element \(0\in\Z\)
			s�dant att f�r alla \(a\in\Z\) g�ller att \(0+a = a+0 = a\).
		\item[Additiv invers] F�r alla \(a\in\Z\) finns ett element \(-a\in\Z\)
			s�dant att \(a + (-a) = (-a) + a = 0\).
	\end{description}
	F�r multiplikation g�ller f�ljande:
	\begin{description}
		\item[Kommutativitet] \(a \cdot b=b \cdot a\) f�r alla \(a,b\in\Z\).
		\item[Associtivitet] \((a \cdot b) \cdot c=a \cdot (b \cdot c)\) f�r
			alla \(a,b,c\in\Z\).
		\item[Multiplikativt identitetselement] Det finns ett element
			\(1\in\Z\) s�dant att f�r alla \(a\in\Z\) g�ller att
			\(1 \cdot a = a \cdot 1 = a\).
%		\item[Multiplikativ invers] F�r alla \(0 \neq a\in\R\) finns ett
%			element \(1/a\in\R\) s�dant att
%			\(a \cdot (1/a) = (1/a) \cdot a = 1\).
	\end{description}
	Ut�ver detta g�ller �ven
	\begin{description}
		\item[Multiplikativ distributivitet �ver addition]
			\(a \cdot (b+c) = (a \cdot b) + (a \cdot c)\) och
			\((b+c) \cdot a = (b \cdot a) + (c \cdot a)\) f�r alla reella tal
			\(a,b,c\in\Z\).
	\end{description}
\end{definition}



%%%%%%%%%%%%%%%%%%%%%%%%%%%%%%%%%%%%%%%%%%%%
% N�GRA SATSER F�R HELTALEN
%%%%%%%%%%%%%%%%%%%%%%%%%%%%%%%%%%%%%%%%%%%%
%\section{N�gra satser f�r heltalen}
%\noindent
%...



%%%%%%%%%%%%%%%%%%%%%%%%%%%%%%%%%%%%%%%%%%%%
% PRIMTAL OCH DELBARTHET
%%%%%%%%%%%%%%%%%%%%%%%%%%%%%%%%%%%%%%%%%%%%
\section{Primtal och delbarhet}
\nocite{Laksov2005kou}
\noindent
...

%%%%%%%%%%%%%%%%%%%%%%%%%%%%%%%%%%%%%%%%%%%%
% DELBARTHET
%%%%%%%%%%%%%%%%%%%%%%%%%%%%%%%%%%%%%%%%%%%%
\subsection{Delbarhet}
\noindent
...

\begin{definition}[Delbarhet]
	...
\end{definition}

\begin{theorem}[Divisionsalgoritmen]
	...
\end{theorem}

\begin{exercise}
	\label{xrc:KvotenMindre}
	Bevisa f�ljande.
	L�t \(a\) och \(b\) vara tv� heltal.
	Kvoten \(q=a/b\) �r strikt mindre �n \(a\) om \(b>1\) �r strikt st�rre �n
	ett.
\end{exercise}


%%%%%%%%%%%%%%%%%%%%%%%%%%%%%%%%%%%%%%%%%%%%
% PRIMTAL OCH SAMMANSATTA TAL
%%%%%%%%%%%%%%%%%%%%%%%%%%%%%%%%%%%%%%%%%%%%
\subsection{Primtal och sammansatta tal}
\noindent
...

\begin{definition}[Primtal och sammansatta tal]
	...
\end{definition}

\begin{theorem}[Aritmetikens fundamentalsats]
	Ett heltal \(n\in\Z\) kan skrivas som en unik produkt av primtal och en
	enhet\footnote{I m�ngden av heltal finns enheterna \(-1\) och \(1\).}.
\end{theorem}
\begin{proof}
	...
\end{proof}

