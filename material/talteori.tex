%%%%%%%%%%%%%%%%%%%%%%%%%%%%%%%%%%%%%%%%%%%%
% PRIMTAL OCH DELBARTHET
%%%%%%%%%%%%%%%%%%%%%%%%%%%%%%%%%%%%%%%%%%%%
\chapter{Talteori}
\label{ch:Talteori}
\nocite{Laksov2005kou}
\noindent
...

%%%%%%%%%%%%%%%%%%%%%%%%%%%%%%%%%%%%%%%%%%%%
% DELBARTHET
%%%%%%%%%%%%%%%%%%%%%%%%%%%%%%%%%%%%%%%%%%%%
\subsection{Delbarhet}
\noindent
...

\begin{definition}[Delbarhet]
	...
\end{definition}

\begin{theorem}[Divisionsalgoritmen]
	...
\end{theorem}

\begin{exercise}
	\label{xrc:KvotenMindre}
	Bevisa f�ljande.
	L�t \(a\) och \(b\) vara tv� heltal.
	Kvoten \(q=a/b\) �r strikt mindre �n \(a\) om \(b>1\) �r strikt st�rre �n
	ett.
\end{exercise}


%%%%%%%%%%%%%%%%%%%%%%%%%%%%%%%%%%%%%%%%%%%%
% PRIMTAL OCH SAMMANSATTA TAL
%%%%%%%%%%%%%%%%%%%%%%%%%%%%%%%%%%%%%%%%%%%%
\subsection{Primtal och sammansatta tal}
\noindent
...

\begin{definition}[Primtal och sammansatta tal]
	...
\end{definition}

\begin{theorem}[Aritmetikens fundamentalsats]
	Ett heltal \(n\in\Z\) kan skrivas som en unik produkt av primtal och en
	enhet\footnote{I m�ngden av heltal finns enheterna \(-1\) och \(1\).}.
\end{theorem}
\begin{proof}
	...
\end{proof}

