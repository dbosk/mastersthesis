%\RequirePackage[swedish]{babel}
%\documentclass[a4paper,reqno,draft,twoside]{amsbook}
\documentclass[a4paper,reqno,final,twoside]{amsbook}

\usepackage{fixltx2e}
\usepackage[latin1]{inputenc}
\usepackage[T1]{fontenc}
\usepackage{graphicx}
\usepackage{url}
\usepackage{hyperref}
\usepackage{lettrine}
\usepackage[swedish]{babel}
%\usepackage[polutonikogreek,swedish]{babel}
%\usepackage[or]{teubner}
\usepackage{amssymb,amsmath,amsthm}
\usepackage{natbib}
%\bibliographystyle{kthnat}
\bibliographystyle{amsnat}
\usepackage{setspace}
\usepackage{nomencl}
\usepackage{marginnote}
%\usepackage{enumerate}
\usepackage{ifdraft}
\ifdraft{
	\usepackage{fancyhdr}
%	\usepackage{showlabels}
	\includeonly{mangder,naturliga,heltalen,talsystem}
}{
	\usepackage[a4paper,rmargin=6cm,lmargin=3cm,bmargin=3.5cm]{geometry}
}

\newtheorem{theorem}{Sats}[chapter]
\newtheorem{lemma}[theorem]{Lemma}
\newtheorem{corollary}[theorem]{Korollarium}
\theoremstyle{definition}
\newtheorem{axiom}[theorem]{Axiom}
\newtheorem{definition}[theorem]{Definition}
\newtheorem{example}[theorem]{Exempel}
\newtheorem{exercise}[theorem]{�vning}
\theoremstyle{remark}
\newtheorem{remark}[theorem]{Anm�rkning}
%\newtheorem*{note}[theorem]{Notera}

\numberwithin{section}{chapter}
\numberwithin{equation}{chapter}

\newcommand{\chref}[1]{Kapitel \ref{#1}}
\newcommand{\secref}[1]{avsnitt \ref{#1}}
\newcommand{\appref}[1]{Appendix \ref{#1}}
\newcommand{\tblref}[1]{Tabell \ref{#1}}
\newcommand{\figref}[1]{Figur \ref{#1}}
\newcommand{\axref}[1]{Axiom \ref{#1}}
\newcommand{\defref}[1]{Definition \ref{#1}}
\newcommand{\thmref}[1]{Sats \ref{#1}}
\newcommand{\lemref}[1]{Lemma \ref{#1}}
\newcommand{\corref}[1]{Korollarium \ref{#1}}
\newcommand{\exref}[1]{Exempel \ref{#1}}
\newcommand{\xrcref}[1]{�vning \ref{#1}}

\newcommand{\imp}[1]{\emph{#1}\marginnote{#1}}
\newcommand{\maybeskip}[1][F�rdjupningsavsnitt.]{\protect\footnote{#1}}

\renewcommand{\qedsymbol}{Q.E.D.}
\newcommand{\N}{\mathbb{N}}
\newcommand{\Q}{\mathbb{Q}}
\newcommand{\R}{\mathbb{R}}
\newcommand{\Z}{\mathbb{Z}}
\newcommand{\powerset}{\mathcal{P}}
\newcommand{\U}{\mathcal{U}}
\newcommand{\V}{\mathcal{V}}
\DeclareMathOperator{\card}{card}
\newcommand{\equivalent}{\Longleftrightarrow}

\makenomenclature
\makeindex

\begin{document}
\ifdraft{
	\fancypagestyle{plain}{
		\renewcommand{\headrulewidth}{0pt}
		\fancyhead{}
		\fancyfoot[co,ce]{\thepage}
		\fancyfoot[ro,le]{dbosk-\today}
	}
	\fancypagestyle{ams}{
		\renewcommand{\headrulewidth}{0pt}
		\fancyhead[ce]{\small\leftmark}
		\fancyhead[co]{\small\rightmark}
		\fancyhead[lo,re]{}
		\fancyhead[ro,le]{\thepage}
		\fancyfoot[ro,le]{dbosk-\today}
	}
	\pagestyle{ams}
}{}
\title{
	%Mathematica prima gamma\\
	Matematik 1c
}
\date{\today}
\author{Daniel Bosk}
\address{Sandgatan 6, 860\,35 S�r�ker}
\email{\url{dbosk@kth.se}}
\urladdr{\url{http://www.bosk.se/}}
%\thanks{...}

\frontmatter
%\begin{abstract}
%    \noindent
%    Dagens matematikundervisning har f�tt mycket kritik p� sistone, bl.a.  fr�n
Skolinspektionen, den upplevs som r�knecentrerad och eleverna l�r sig genom
memorering snarare �n f�rst�else.
Denna negativa utveckling beh�ver v�ndas.

Detta material syftar till att l�gga grunden f�r en formalisering av
gymnasiematematiken, det �r en b�rjan till undervisningsmaterial utformat
efter �mnesplanen f�r Matematik 1c.
Omfattningen av detta material �r en inledning till rigor�s matematik, en grund
att bygga fortsatt undervisning p�.
Syftet med detta material �r s�ledes inte att vara ett f�rdjupningsmaterial,
utan att formalisera dagens gymnasiematematik.
St�d f�r detta finns i den nya �mnesplanen Matematik 1c, som f�reskriver h�gre
krav p� rigor�st matematiskt inneh�ll.

%\end{abstract}

\ifdraft{}{
    \newgeometry{rmargin=4cm,lmargin=4cm}
}
\maketitle
\ifdraft{}{
    \restoregeometry
}
%\begin{titlepage}
%   \begin{center}
%       \vskip0.1\textheight
%       \textsc{\huge \...@title}
%       \vskip0.1\textheight
%       \textsc{\large \...@author}
%       \vskip0.7\textheight
%       \textsc{\...@date}
%   \end{center}
%\end{titlepage}

\printnomenclature
\tableofcontents

\chapter{F�rord}
\noindent
Inspirationen till denna studie har varit min k�rlek f�r matematik och dess
underbara generaliserbarhet samt min pedagogiska �vertygelse att alla kan
l�ra sig matematik.
Min egen erfarenhet fr�n svensk skola och mina universitetsstudier i matematik
s�ger mig att matematiken som den undervisas i grundskola och gymnasieskola �r
l�ngt fr�n matematiken som den undervisas p� universiteten vidare ocks� fr�n
den som forskas p�.
Jag har haft tankar om att den generaliserade matematiken med bevis och
h�rledningar, som den f�rst �r p� universitetsniv�, ger mer f�rst�else till
eleverna �n dagens undervisning i r�kning med den evinnerliga repetitionen
genom r�knande av tal efter tal.
Jag har m�tt m�nga som sett matematiken som en kombination av memorering av hur
siffror ska kombineras i olika situationer och i annat fall gissande av hur de
ska kombineras.
Detta �r ej undervisning d� matematikenlektionerna idag �r utformade f�r att
eleverna utifr�n ett f�tal givna exempel ska f�rs�ka att generalisera kunskaper
som de sedan ska anpassa efter ett st�rre antal testuppgifter.
Jag har ocks�, inom det datalogiska omr�det maskininl�rning, sett algoritmer
som arbetar p� samma s�tt.
�r det d� m�jligt att s�ga att v�ra svenska matematikelever uppn�r en h�gre
f�rst�else f�r inneh�llet �n dessa datorprogram?
%Jag har sj�lv sv�rt att finna en s�dan anledning.
%De skulle f� ungef�r samma resultat p� provet efter�t.

Projektet har tagit b�de tid och kraft men har varit otroligt roligt och
givande.
Jag vill tacka mina handledare Roy Skjelnes och Lil Engstr�m f�r
givande diskussioner kring arbetet och bra �terkoppling.
Jag vill ocks� tacka Dan Laksov som �ven han deltagit och bidragit med
v�rdefulla insikter.
Slutligen vill jag tacka den skola, den l�rare och de elever som tagit sig
tiden att delta i unders�kningen.

\vspace{2cm}
\begin{flushright}
	\parbox{5cm}{
	S�r�ker, juni 2011\\
	Daniel Bosk
	}
\end{flushright}


\mainmatter
\ifdraft{
	\doublespacing
}{}
% main chapters
\chapter{Inledning}
\noindent
Dagens matematikundervisning har f�tt mycket kritik, bland annat
fr�n \citet{Skolinspektionen2010uim}.
Matematiken i grundskole- och gymnasieundervisningen upplevs som r�knecentrerad
och eleverna l�r sig genom memorering snarare �n
f�rst�else~\citep{Skolinspektionen2010uim,Johansson2006tmw}.
Skolinspektionens kvalitetsgranskningsrapport \citep{Skolinspektionen2010uim}
om matematikundervisningen i gymnasieskolan tar ocks� upp att
matematikkunskaperna hos svenska elever med �ren blivit s�mre, rapporten
h�nvisar till bland annat \emph{TIMSS Advanced 2008}.
Detta m�rks ocks� av p� h�gskolan.
\citet{Thunberg2006gmo} skriver om att gapet mellan gymnasiematematiken och
h�gskolematematiken blivit st�rre och att studenterna inte har de f�rkunskaper
som h�gskolan f�rv�ntar sig n�r de forts�tter att studera.
�ven min egen erfarenhet i svensk skola och universitetsstudier i matematik
s�ger mig att matematiken som den undervisas i grundskola och gymnasium �r
l�ngt fr�n matematiken som den undervisas p� universiteten och den som forskas
p�.

%Mitt eget intresse f�r matematik tog fart i grundskolan vid introduktionen av
%algebra.
%Jag fick d� en insikt i matematikens generaliserbarhet och dess koppling till
%programmeringen, som d� var ett nyfunnet intresse.
%Jag fortsatte att studera matematik genom gymnasiet och vidare till h�gskolan.

Hur skiljer sig d� grundskole- och gymnasiematematiken fr�n den p�
universiteten och h�gskolorna?
\citet{Agahi2010mdi} har unders�kt hur ett gymnasie- och ett h�gskolel�romedel
i matematik skiljer sig med avseende p� matematiska definitioner.
Definitioner �r mycket grundl�ggande inom matematiken, det �r i definitionerna
som de olika matematiska begreppens betydelse och egenskaper fastst�lls.
Det �r s�ledes i dessa som alla matematiska resonemang tar sin utg�ngspunkt.
Unders�kningen visade att de tv� l�romedlen har ungef�r liknande m�ngd
definitioner men att de skiljde sig i vilken typ av definitioner de inneh�ll.
Gymnasielitteraturen tenderade till att i stor del beskrivas genom exempel
och i n�gra fall med exempel i kombination med att ange egenskaper.
H�gskolel�romedlet tenderade att fokusera p� egenskaper i definitionerna och i
vissa fall �ven ge exempel.

Varf�r skiljer sig d� grundskole- och gymnasiematematiken fr�n den p�
universiteten?
Matematikhistoriskt (j�mf�r \citet{Kline1990mtf1,Kline1990mtf2,Kline1990mtf3})
har matematik l�nge varit den typ av matematik som studeras vid universiteten,
det vill s�ga \emph{formell} eller \emph{rigor�s matematik} med definitioner,
satser och krav p� bevis.
Matematiker har str�vat efter denna rigorositet �tminstone sedan 
%Definitioner, satser och bevis har varit en del av matematiken �tminstone sedan
Euklides (omkring 300 f.Kr.) skrev sina 13 volymer av \emph{Elementa}, �ven om
de inte alltid varit lika formella.
Kraven p� exakta definitioner, satser och bevis som de �r inom matematisk
forskning idag kan s�gas n�dde dagens niv� under matematikens formella
grundande under 1800-talet \citep{Kline1990mtf3}.
F�ljdaktligen kan det inte vara att universiteten nyligen b�rjat med en ny typ
av matematik och grund- och gymnasieskolorna inte har hunnit med.
En unders�kning \citep{Johansson2006tmw} av ett popul�rt svenskt
matematikl�romedels utveckling fr�n 1979 �rs upplaga fram till 2001 �rs upplaga
visar att dessa tv� upplagor har mycket gemensamt, trots att de skiljs �t med
tre l�roplansreformer.
Detta �r intressant d� den f�rsta boken borde vara anpassad efter l�roplanen
Lgr 69 och den sista efter l�roplanen Lpo 94, och d�remellan kom ocks� Lgr 80.
Det som skiljer b�ckernas inneh�ll �r olika tematiska avsnitt och
probleml�sningsavsnitt.
Unders�kningen visar ocks� att dessa avsnitt vanligtvis bara anv�nds i m�n av
tid och att eleverna tillbringar en stor del av tiden till egen r�kning i boken.
F�ljdaktligen skulle tradition m�jligen kunna ses som en del av anledningen
till dagens skillnad mellan h�gskolematematiken och �vriga svenska
utbildningsformer, men det �r sv�rt att dra n�gon slutsats av enbart detta.
D�rf�r blickar detta arbete fram�t.

%Kan det vara att eleverna m�ste uppn� en viss mognadsgrad f�r att kunna ta till
%sig denna typ av matematik?
%%Av matematikhistorien att d�ma �r svaret p� fr�gan ''Nej''.
%Om man ser till historiens stora matematikers genombrotts�lder �r den omkring
%tjugo�rs�ldern.
%Det motsvarar den �lder n�r en svensk gymnasieelev tar studenten.
%Det verkar rimligt att antaga att vid genombrotts�ldern har dessa personer
%redan studerat denna typ av matematik under en tid, och s�ledes har de studerat
%denna typ av matematik under den motsvarande gymnasietiden.
%%Tv� stora matematiker, �variste Galois (1811--1832) och Nils Henrik Abel
%%(1802--1829), dog vid 20 respektive 26 �rs �lder.
%Det verkar inte finnas n�gon tydlig anledning till skillnaderna mellan
%gymnasiet och universitetet, jag har inte funnit n�gon forskning tydligt ger en
%anledning eller motivering.

Till h�stterminen 2011 f�r Sverige en ny gymnasieskola och det �r �terigen dags
att anpassa undervisningsmaterial och l�romedel.
De gamla programmen som alla gav h�gskolebeh�righet ers�tts med yrkes- och
studef�rberedande program, d�r endast de senare ger h�gskolebeh�righet
\citep{Skolverket2011gy}.
Yrkesprogrammen ska dock tillhandah�lla kurser som individuellt val som ger
h�gskolebeh�righet.
I samband med detta ers�tts ocks� den gamla kursplanen Matematik A
\citep{SKOLFS2000-5} med en ny �mnesplan f�r matematik och kurserna Matematik
1a,
Matematik 1b och Matematik 1c \citep{SKOLFS2010-261}.
Kursen Matematik 1a v�nder sig till de yrkesf�rberedande programmen, Matematik
1b till bland annat det studief�rberedande Samh�llsvetenskapliga programmet
medan Matematik 1c v�nder sig till de studief�rberedande programmen
Teknikprogrammet och Naturvetenskapsprogrammet.
D� den f�rsta kursen i matematik inte l�ngre �r gemensam f�r alla
gymnasieelever, utan ges i olika versioner beroende p� program, kan
matematikundervisningen f�r den f�rsta kursen till�tas att variera mer.
Detta �ppnar f�r id�n om att �tminstone eleverna i m�lgruppen f�r Matematik 1c,
de som �mnar att l�sa vidare, kan undervisas en mer formell matematik.
Inneh�llet i de olika versionerna av kurserna �r till grunden detsamma, men de
skiljer sig i hur djupt de g�r i de olika delomr�dena.
Till exempel kan vi titta p� omr�det \emph{Taluppfattning, aritmetik och
algebra} f�r de b�da kurserna Matematik 1a och Matematik 1c.
Inneh�llet i detta omr�de f�r Matematik 1a �r enligt
\citet[sidan 89]{SKOLFS2010-261}:
\begin{quote}
	\begin{itemize}
		\item Metoder f�r ber�kningar med reella tal skrivna p� olika former
			inom vardagslivet och k�rakt�rs�mnena, inklusive �verslagsr�kning,
			huvudr�kning och uppskattning samt strategier f�r att anv�nda
			digitala verktyg.
		\item Strategier f�r att anv�nda hj�lpmedel fr�n karakt�rs�mnena, till
			exempel formul�r, mallar, tumregler, f�reskrifter, manualer och
			handb�cker.
		\item Hantering av algebraiska uttryck och f�r karakt�rs�mnena
			relevanta formler samt metoder f�r att l�sa linj�ra ekvationer.
	\end{itemize}
\end{quote}
Inneh�llet f�r Matematik 1c formuleras d�remot som f�ljande \citep[sidan
101]{SKOLFS2010-261}:
\begin{quote}
	\begin{itemize}
		\item Egenskaper hos m�ngden av heltal, olika talbaser samt begreppen
			primtal och delbarhet.
		\item Metoder f�r ber�kningar inom vardagslivet och karakt�rs�mnena med
			reella tal skrivna p� olika former, inklusive potenser med reella
			exponenter samt strategier f�r anv�ndning av digitala verktyg.
		\item Generalisering av aritmetikens r�knelagar till att hantera
			algebraiska uttryck.
		\item Begreppet linj�r olikhet.
		\item Algebraiska och grafiska metoder f�r att l�sa linj�ra ekvationer
			och olikheter samt potensekvationer.
	\end{itemize}
\end{quote}
Skillnaderna best�r i att inneh�llet i Matematik 1a fokuserar p� praktisk
r�kning med huvudr�kning, �verslagsr�kning, tumregler och relevanta formler,
medan Matematik 1c �r mer teoretiskt inriktad med talteori, talbaser och
potenser med reella exponenter.
I Matematik 1a �r det linj�ra ekvationer medan i Matematik 1c �r det linj�ra
ekvationer, olikheter och potensekvationer.

Under omr�det \emph{Geometri} i det centrala inneh�llet f�r Matematik 1c
\citep[sidan 101]{SKOLFS2010-261} finns f�ljande tv� punkter med\footnote{Det
kan p�pekas att dessa tv� punkter �ven finns med i Matematik 1b med texten
''inom naturvetenskapliga �mnen'' utbytt mot ''inom olika �mnesomr�den''.
Punkterna finns d�remot inte med under centralt inneh�ll f�r Matematik 1a.}:
\begin{quote}
	\begin{itemize}
		\item Matematisk argumentation med hj�lp av grundl�ggande logik
			inklusive implikation och ekvivalens samt j�mf�relser med hur man
			argumenterar i vardagliga sammanhang och inom naturvetenskapliga
			�mnen.
		\item Illustration av begreppen definition, sats och bevis, till
			exempel med Pythagoras sats och triangelns vinkelsumma.
	\end{itemize}
\end{quote}
Enligt den nya gymnasief�rordningen \citep[1 kapitlet 4 \S]{SFS2010-2039} ska
inte det centrala inneh�llet i �mnesplanerna vara en begr�nsning utan utrymme
ska ges till l�rare och elever f�r att tillsammans utforma undervisningen.
Det vill s�ga, uppdelningen av en kurs centrala inneh�ll beh�ver inte p�verka
undervisningens uppl�gg.
Dessutom kan inneh�ll ut�ver det centrala inneh�llet l�ggas till om det beh�vs.
Detta g�r punkterna givna ovan kan genomsyra hela undervisningen ist�llet f�r
att begr�nsas till ett geometriavsnitt.
Fr�gorna som �terst�r �r s�ledes vilket inneh�ll som beh�vs och om det �r
m�jligt att t�cka det centrala inneh�llet med formell matematik inom kursens
tidsramar och om det �r m�jligt att undervisa inneh�llet med formell matematik
f�r gymnasieelever.


\section{Syfte och fr�gest�llning}
\noindent
Detta projekts huvudsyfte �r att l�gga grunden f�r en formalisering av
gymnasiematematiken, genom att ta fram en b�rjan till undervisningsmaterial
motsvarande inneh�llet f�r nya gymnasieskolans Matematik 1c samt att se om och
hur denna typ av matematik kan undervisas i skolan.
Fokus ligger p� att ta fram ett kompendium med en inledning till rigor�s
matematik, en grund att bygga fortsatt undervisning p�, f�r att se hur mycket
som kr�vs inneh�llsm�ssigt f�r att inom �mneplanens ramar undervisa formell
matematik.
Med rigor�s och formell matematik avses matematik v�l grundad i logiskt
resonemang med exakta definitioner och satser bevisade utifr�n dessa.

De fr�gest�llningar som ligger till grund f�r detta arbete �r f�ljande:
\begin{enumerate}
	\item\label{rq:Till�tande}
		Hur till�ter, eller hindrar, gymnasieskolans nya �mnesplaner i
		matematik att undervisa formell matematik?
	\item\label{rq:Material}
		Hur kan ett undervisningsmaterial f�r formell matematik motsvarande
		Matematik 1c se ut?
	\item\label{rq:M�jligt}
		Kan formell matematik undervisas p� gymnasieniv�?
\end{enumerate}

Det har redan konstaterats ovan att fr�ga \ref{rq:Till�tande} delvis redan
besvarats, juridiskt �r det till�tet att undervisa formell matematik.
En del som inte besvarats �r dock huruvida det �r m�jligt inom tidsramarna
eller om det kr�vs f�r mycket omkringliggande teori f�r att kunna behandla
hela kursens centrala inneh�ll p� detta s�tt.
Tyv�rr kan inte hela denna fr�ga besvaras inom ramarna f�r detta arbete d� det
skulle inneb�ra att utveckla ett l�romedel f�r hela Matematik 1c.
Att utveckla en inledning till ett s�dant borde dock �nd� ge en viss
uppskattning av hur mycket extra inneh�ll som kr�vs.
En ytterligare del av fr�ga \ref{rq:Till�tande} �r vilken typ av
�vningsuppgifter som �mnesplanen kr�ver.
�mnesplanen f�r matematik \citep{SKOLFS2010-261} talar om olika f�rm�gor som
eleverna ska f� m�jlighet att utveckla.

F�r att precisera vilka delar av fr�ga \ref{rq:Till�tande} som detta arbete ska
f�rs�ka att besvara, kan fr�gan ist�llet delas upp i f�ljande delfr�gor.
\begin{enumerate}
	\item[\ref{rq:Till�tande}.]
		Hur till�ter, eller hindrar, gymnasieskolans nya �mnesplaner i
		matematik att undervisa formell matematik?
		\begin{enumerate}
			\item\label{rq:M�ngdInneh�ll}
				Hur mycket extra inneh�ll beh�vs uppskattningsvis f�r att
				undervisa Matematik 1c med formell matematik?
			\item\label{rq:�vningsuppgifter}
				Vilken typ av �vningsuppgifter kr�ver �mnesplanen?
		\end{enumerate}
\end{enumerate}

\chapter{Logik och bevis}
\label{ch:Logik}
\nocite{Bartle2000itr}
\noindent
\lettrine{M}{atematiken har sin} grund i logiken.
Det �r logiken som ger matematiken m�jligheten till ett resonemang och
m�jligheten till h�rledning.
Matematiken utg�r fr�n n�gra f� grundl�ggande antaganden, kallade \emph{axiom},
fr�n vilka alla matematiska resultat h�rleds.
En matematiker n�jer sig allts� inte med att unders�ka n�gra exempel -- eller
g�ra experiment -- inte ens tusen- eller miljontals exempel duger.
Det �r detta som skiljer matematiken fr�n exempelvis fysiken och kemin, trots
att dessa till mycket stor omfattning anv�nder matematiken som hj�lpmedel.
Vi har inga grundl�ggande principer f�r fysiken och kemin som vi k�nner till,
men vi har det f�r matematiken.
Vi har det f�r matematiken f�r att den �r skapad av oss.


%%%%%%%%%%%%%%%%%%%%%%%%%%%%%%%%%%
% LOGIK
%%%%%%%%%%%%%%%%%%%%%%%%%%%%%%%%%%
\section{Logik}
\noindent
\lettrine{A}{ll matematisk argumentation} best�r av \emph{utsagor}.
Dessa �r deklarativa meningar som kan klassificeras som antingen \emph{sanna}
eller \emph{falska}.
Vi beh�ver inte alltid veta precis vilket, men det m�ste vara den ena eller den
andra -- aldrig b�da.
Detta kallas \emph{Lagen om det uteslutna tredje}.
Om vi tittar p� f�ljande meningar:
\begin{enumerate}
	\item Denna text �r skriven p� svenska.
	\item Gr�nt �r en fin f�rg.
	\item Denna mening �r falsk.
	\item Det finns o�ndligt m�nga tvillingprimtal.
	\item \(x^2+1=0\)
\end{enumerate}
Den f�rsta meningen �r en utsaga, och den �r sann.
Den andra meningen �r ej en utsaga i logisk bem�rkelse, det �r en smaksak.
Den tredje meningen �r ej en utsaga, den kan varken vara sann eller falsk
eftersom att det leder till mots�gelsefulla slutsatser.
Den fj�rde meningen �r en utsaga, ingen vet dock om den �r sann eller falsk.
Den femte och sista symbolf�ljden �r en utsaga, men vi vet inte vad \(x\) �r s�
vi kan inte uttala oss om den skulle vara sann eller falsk.
Detta visar vikten av att tydligt specificera alla delar av en utsaga, s� att
det �r alldeles klart vad vi menar.
Den femte utsagan skulle allts� beh�va �ndras till exempelvis ''det finns ett
komplext tal \(x\) s�dant att \(x^2+1=0\)'', den skulle d� vara sann.
Om vi ist�llet �ndrat den till ''det finns ett heltal \(x\) s�dant att
\(x^2+1=0\)'' skulle den vara falsk oavsett hur vi v�ljer \(x\) eftersom att
det inte finns ett s�dant heltal.  
En utsaga som alltid �r falsk kallar vi f�r \emph{mots�gelse} eller
\emph{kontradiktion}.
En utsaga som alltid �r sann kallar vi f�r \emph{tautologi}.

Tv� utsagor \(P\) och \(Q\) s�gs vara logiskt ekvivalenta om \(P\) �r sann
precis n�r \(Q\) �r sann och allts� om \(P\) �r falsk precis n�r \(Q\) �r falsk.
Vi skriver detta som \(P\equiv Q\).

\subsection{Kombinerade utsagor}
\noindent
Vi vill ocks� kunna forma nya utsagor fr�n redan k�nda, detta genom att
kombinera och modifiera dem.
Om \(P\) �r en utsaga, d� s�ger vi att \emph{negationen} av \(P\), betecknad
\(\lnot P\) eller \(not P\), �r falsk precis n�r \(P\) �r sann och sann precis
n�r \(P\) �r falsk.
Vi kommer d� fram till \emph{Lagen om dubbelnegation}.
Om vi funderar p� vad som h�nder om vi tar \(\lnot(\lnot P)\) s� kommer vi
fram till att \(P\equiv \lnot(\lnot P)\).
Ett exempel p� negation, l�t \(P\) vara utsagan ''vi befinner oss i Sverige''.
D� blir \(\lnot P\) ''vi befinner oss ej i Sverige''.
Vi kan ocks� titta p� f�ljande utsaga, ''alla svenskar tycker om
surstr�mming''.
Negationen av den utsagan �r \emph{inte} att ingen svensk tycker om
surstr�mming, den �r ''det finns n�gon svensk som inte tycker om
surstr�mming'' -- detta �r en v�ldigt viktig skillnad att inte g�ra fel p�!

Vi kan ocks� kombinera utsagor genom \emph{konjunktioner}.
Om \(P\) och \(Q\) �r utsagor, d� betecknar vi konjunktionen som \(P och Q\)
eller \(P\land Q\).
Konjunktionen �r sann d� b�de \(P\) och \(Q\) b�da �r sanna och falsk annars.
Vi har ocks� \emph{disjunktionen} som betecknas \(P eller Q\) eller \(P\lor
Q\).
Disjunktionen �r sann om antingen \(P\) eller \(Q\) eller b�da �r sanna, och �r
s�ledes falsk endast n�r \(P\) och \(Q\) b�da �r falska.
Konjunktionen och disjunktionen sammanfattas i en sanningstabell i
\tblref{tbl:SanningKonjunktionDisjunktion}.

\begin{table}[h]
	\caption{Sanningstabell f�r konjunktionen och disjunktionen.}
	\begin{tabular}{c|c|c|c}
		\hline\hline
		\(P\)		& \(Q\)			& \(P\land Q\)	& \(P\lor Q\) \\
		\hline
		T			&	T			& T 			& T \\
		T			&	F			& F				& T \\
		F			&	T			& F				& T \\
		F			&	F			& F				& F \\
		\hline\hline
	\end{tabular}
	\label{tbl:SanningKonjunktionDisjunktion}
\end{table}

\begin{exercise}
	Testa att kombinera negationen, konjunktionen och disjunktionen, g�r det
	att forma n�gra logiskt ekvivalenta uttryck?
\end{exercise}


\subsection{Implikationer}
\noindent
Implikation �r synonymt med ordet \emph{medf�r}.
Om \(P\) och \(Q\) �r utsagor s�ger vi att \(P\) \emph{implicerar} \(Q\) eller
\emph{om \(P\), d� \(Q\)}.
Vi ska unders�ka n�r denna sammansatta utsaga b�r vara sann och n�r den b�r
vara falsk.
L�t oss formulera ett exempel, titta p� utsagan ''Om jag vinner pengar, d� ska
jag k�pa nya b�cker till skolan''.
Den �r uppenbart falsk om jag vinner pengar men inte k�per b�cker till skolan,
men sann om jag g�r det.
Annars, om jag inte vinner pengar, d� har jag heller inte lovat att k�pa b�cker
till skolan.
Allts� m�ste utsagan vara sann i det fallet.
Men att jag inte vinner pengar hindrar mig ju inte att k�pa b�cker till skolan
�nd�, allts� borde utsagan vara sann �ven i det fallet.
Implikationens olika sanningsv�rden sammanfattas i
\tblref{tbl:SanningImplikation}.

Vi kan naturligtvis v�nda p� implikationen, om \(P\) och \(Q\) �r utsagor och
\(P\implies Q\) d� s�ger vi att dess \emph{omv�ndning} �r \(Q\implies P\).
Omv�ndningen f�r en implikation �r inte n�dv�ndigtvis logiskt ekvivalent med
implikationen.
Ett exempel f�r illustrera, ''Om vi �r i Stockholm, d� �r vi i Sverige'' �r en
sann utsaga.
Dess omv�ndning ''Om vi �r i Sverige, d� �r vi i Stockholm'' �r d�remot inte
sann eftersom att vi skulle kunna vara i exempelvis Sundsvall, G�teborg eller
Kiruna som ocks� �r st�der i Sverige.
Om vi d�remot tittar p� utsagan ''Om vi inte �r i Sverige, d� �r vi inte i
Stockholm''.
Denna utsaga �r sann, men hur f�rh�ller den sig till den ursprungliga
implikationen?
Jo, om \(P\) f�r vara utsagan ''vi �r i Stockholm'' och \(Q\) f�r vara ''vi �r
i Sverige'', d� var den ursprungliga implikationen \(P\implies Q\) och den
sista sanna utsagan �r \(\lnot Q\implies\lnot P\).
Denna kallas f�r den \emph{kontrapositiva} utsagan.
Men om \(P\implies Q\) och \(Q\implies P\) b�da skulle vara sanna, d� skriver
vi detta som \(P\equivalent Q\).
Utsagan \(P\equivalent Q\) kallas f�r \emph{dubbelimplikation} eller
\emph{ekvivalens} och �r sann d� \(P\) och \(Q\) b�da �r sanna och d� de b�da
�r falska.
Den utl�ses som \emph{\(P\) om och endast om \(Q\)}.

Vi ska nu avsluta med en viktig logisk ekvivalens till implikationen.
Denna ligger till grund f�r mots�gelsebevis.
Den s�ger att \((P\land\lnot Q)\implies C\), d�r \(C\) �r en mots�gelse och
allts� alltid �r falsk, �r logiskt ekvivalent med \(P\implies Q\).
Detta ses tydligast i en sanningstabell, och den �r given i
\tblref{tbl:SanningImplikation}.

\begin{table}[t]
	\caption{Sanningstabell f�r implikationen och dess logiskt ekvivalenta
		former.}
	\begin{tabular}{c|c|c|c|c|c|c}
		\hline\hline
		\(P\) & \(Q\) & \(P\implies Q\)	& \(\lnot(P\land \lnot Q)\) &
			\(\lnot Q\implies \lnot P\) &
			\(C\) & \((P\land\lnot Q)\implies C\) \\
		\hline
		T & T & T & T & T & F & T \\
		T & F & F & F & F & F & F \\
		F & T & T & T & T & F & T \\
		F & F & T & T & T & F & T \\
		\hline\hline
	\end{tabular}
	\label{tbl:SanningImplikation}
\end{table}


%%%%%%%%%%%%%%%%%%%%%%%%%%%%%%%%%%
% AXIOM
%%%%%%%%%%%%%%%%%%%%%%%%%%%%%%%%%%
\section{Axiom}
\noindent
\lettrine{F}{�r att det} ska kunna g� att h�rleda n�gonting m�ste det finnas
n�gra grundl�ggande utsagor som en grund att bygga p�.
Dessa utsagor kallar vi f�r \emph{axiom}\index{axiom}, och de �r fr�n dessa
alla matematiska h�rledningar utg�r.
Vi har ocks� definitioner som indirekt kan specificera axiom.
Axiomen kan ej h�rledas eftersom att de �r startpunkten f�r all h�rledning.

Vi har ovan redan sett n�gra axiom, n�mligen de logiska axiomen.
Vi var dock inte tydliga med detta eftersom att vi inte visste vad ett axiom
var.
Vi ska i kommande kapitel ta upp de matematiska axiomen.
Det finns axiom som g�ller f�r hela matematiken, dessa �r axiomen f�r
m�ngdl�ran, och det finns olika axiomupps�ttningar inom specifika omr�den inom
matematiken.
Axiomen f�r m�ngdl�ran ligger dock utanf�r m�len f�r denna text, vi kommer att
n�ja oss med en definition av begreppet m�ngd och sedan utg� fr�n en annan
upps�ttning axiom, Peanos axiom f�r de naturliga talen, som duger f�r v�ra
�ndam�l.


%%%%%%%%%%%%%%%%%%%%%%%%%%%%%%%%%%
% BEVIS
%%%%%%%%%%%%%%%%%%%%%%%%%%%%%%%%%%
\section{Bevis}
\noindent
\lettrine{G}{runden m� vara} viktig att st� p�, men m�jligheten att ta oss
vidare till nya resultat -- bevis -- �r ocks� av yttersta vikt.
Faktum �r ju att vi tidigare beh�vde logiken f�r att kunna resonera och dra
slutsatser, f�r att kunna bevisa nya resultat.
I detta avsnitt kommer vi att g� igenom n�gra vanliga bevismetoder.

N�r ett bevis genomf�rs och presenteras brukar detta avslutas med 
\textsc{Q.E.D.} som �r en f�rkortning f�r latinets \emph{Quod Erat
Demonstrandum} och betyder \emph{vilket skulle visas}.

\subsection{Motexempelbevis}
\noindent
Vi b�rjar med den enklaste bevismetoden.
Om n�gon skulle p�st� att ''alla svenskar tycker om surstr�mming'', d� r�cker
det med att vi hittar en svensk som inte tycker om surstr�mming f�r att
motbevisa p�st�endet.
Det vill s�ga, vi hittar ett motexempel.
Kom ih�g fr�n tidigare att negationen av utsagan ''alla svenskar tycker om
surstr�mming'' �r ''det finns n�gon svensk som inte tycker om surstr�mming''
och att det �r denna utsaga som vi bevisar genom att finna en s�dan svensk.

\subsection{Direkta bevis}
\noindent
Vi l�ter \(P\) och \(Q\) vara utsagor.
F�r att hypotesen \(P\) ska implicera konklusionen \(Q\) m�ste \(P\) vara sann
precis n�r \(Q\) �r sann.
Vi �stadkommer detta genom konstruktionen av en kedja av implikationer
\[
	P\implies R_1, R_1\implies R_2, \ldots, R_n\implies Q.
\]
Enligt \emph{Lagen om syllogism} m�ste d� \(P\implies Q\).
Karakteristiskt f�r denna bevismetod �r att det bara �r att ''r�kna p�'' s�
kommer vi fram till konklusionen.

\subsection{Kontrapositiva bevis}
\noindent
L�t \(P\) och \(Q\) vara utsagor.
Eftersom att vi tidigare, i \tblref{tbl:SanningImplikation}, sett att den
kontrapositiva implikationen \(\lnot Q\implies\lnot P\) �r logiskt ekvivalent
med \(P\implies Q\) kan vi likv�l bevisa den kontrapositiva implikationen som
\(P\implies Q\).
Vi vill kunna g�ra detta f�r att detta ibland kan vara l�ttare �n att visa att
\(P\) medf�r \(Q\).

\subsection{Mots�gelsebevis}
\noindent
Mots�gelsebeviset �r kanske den bevismetod som anv�nds flitigast i detta
kompendium.
Metoden �r v�ldigt effektiv och kan ofta vara enklare att anv�nda �n att
konstruera ett direkt bevis.
Den anv�nder, precis som f�reg�ende metod, att \((P\land\lnot Q)\implies C\) �r
logiskt ekvivalent med \(P\implies Q\) n�r \(C\) �r en mots�gelse.
Den s�ger att vi ska vi ska anta v�r hypotes \(P\) och �ven anta motsatsen
\(\lnot Q\) till v�r �nskade konklusion \(Q\).
Om dessa antaganden tillsammans leder till en utsaga som alltid �r falsk, det
vill s�ga en mots�gelse \(C\), d� har vi visat att \(P\) implicerar \(Q\)
eftersom att dessa �r logiskt ekvivalenta.

%\subsection{Induktionsbevis}
%\noindent
%...

\chapter{M�ngder}
\noindent
...

\begin{definition}
	...
\end{definition}

%\chapter{Funktioner\maybeskip[Kapitlet �r med enbart f�r att visa
dispositionen.]}
\label{ch:Funktioner}
\noindent
...

\begin{definition}[Funktion]
	...
\end{definition}

\chapter{De naturliga talen}
\label{ch:Naturliga}
\nocite{Kline1990mtf3}
\noindent
\lettrine{D}{e naturliga talen} �r de tal som vi anv�nder f�r att ordna
och r�kna saker.
Talen \(1,2,3,\ldots\) �r naturliga tal och det �r dessa tal som m�nniskan
anv�nt l�ngst i historien.
Talet noll kom v�ldigt sent i historien, s� sent som p� 800-talet i Indien,
medan de andra talen anv�nts sedan �rtusenden tillbaka i tiden.
Vi ska �nd� ta med talet \(0\) bland v�ra naturliga tal.
M�ngden av naturliga tal betecknas \(\N\), det vill s�ga
\(\N=\{0,1,2,3,\ldots\}\).
D� dessa tal funnits l�nge i m�nniskans historia har de ocks� betraktats som
sj�lvklara, sj�lvklara att den tyske matematikern Leopold Kronecker (1823-1891)
sade ''\foreignlanguage{german}{Die ganzen Zahlen hat der liebe Gott gemacht,
alles andere ist Menschenwerk}'', p� svenska ''Den k�re Gud har skapat de hela
talen, allt annat �r m�nniskans verk.''

Under 1800-talet blev dock matematikerna uppm�rksamma p� att det beh�vdes en
stadigare grund att bygga matematiken p�.
Det gick inte l�ngre att anta talen som sj�lvklara.
Hittills hade de naturliga talen (detta kapitel), heltalen (\chref{ch:Heltalen})
och de reella talen (\chref{ch:Reella}) ansetts vara sj�lvklara.
Men nu b�rjade sj�lvklarheten att ifr�gas�ttas.

I detta kompendium kommer grunden att l�ggas f�rst och sedan forts�tter vi
v�r v�g upp�t.
Det vill s�ga, vi b�rjar med de naturliga talen och g�r sedan vidare till de
hela talen, de rationella talen och slutligen de reella talen.
Att d�ma av det historiska f�rloppet grundades matematiken egentligen i omv�nd
ordning.
Det vill s�ga, de reella talen grundades f�rst.
Sedan se rationella talen, de hela talen och sist de naturliga talen.
%De reella talen var f�rst att grundas, de grundades p� de rationella talen.
%De rationella talen grundades d�refter p� heltalen.
%Heltalen grundades d�refter p� de naturliga talen.
%Slutligen grundades de naturliga talen genom de axiom som tas upp i detta
%kapitel.
%Analogt kan s�gas att taket p� huset byggdes innan grunden var lagd och
%v�ggarna resta.
Vi ska nu i kommande avsnitt titta n�rmare p� de axiom som ligger till grund
f�r de naturliga talen.


\section[Peanos axiom f�r de naturliga talen]
	{Peanos axiom f�r de naturliga talen\maybeskip}
\noindent
\lettrine{U}{nder 1800-talets} slut och b�rjan av 1900-talet grundades de delar
av matematiken som redan anv�nts sedan �rtusenden tillbaka.
De naturliga talen fick sin axiomatiska grund n�r Richard Dedekind (1831-1916)
�r 1888 publicerade ett antal axiom f�r de naturliga talen.
�ret efter publicerade dock Giuseppe Peano (1858-1932) en f�rb�ttring av dessa
axiom och det �r Peanos f�rb�ttrade axiom som godtagits och anv�nds idag,
om �n i lite annorlunda formulering.

%Vi ska �ven se i \secref{sec:vonNeumannNaturliga} att Peanos axiom
%faktiskt kan h�rledas ur m�ngdteorin och att m�ngdteorin d�rf�r kan anv�ndas
%som fundamental teori f�r matematiken.
%Dedekind och Peanos axiom kan �nd� anv�ndas som grund f�r de naturliga talen,
%men ist�llet f�r att vara axiom blir de satser som kan h�rledas i m�ngdteorin.

Vi ska nu titta p� Peanos axiom f�r de naturliga talen.
B�rja med att sl�ppa taget om allt du tror dig k�nna till om matematiken.
N�r du forts�tter efter denna mening ska din ''matematikv�rld'' vara helt tom.
D�refter kan du fylla den med axiomen alltefter de presenteras i texten.
Vi b�rjar nu med det f�rsta axiomet.
\begin{axiom}
	\label{ax:NaturligaAxiomNoll}
	\(0\) �r ett naturligt tal. 
\end{axiom}
Det f�rsta axiomet, \axref{ax:NaturligaAxiomNoll}, s�ger helt enkelt att det
finns �tminstone ett naturligt tal.
Det s�ger ingenting mer om \(0\) �n att det �r ett naturligt tal och
vi vet inte �nnu vilka egenskaper som nollan besitter.
Detta �r allt som nu finns i v�r matematikv�rld -- ''noll �r ett naturligt
tal''.
Vi g�r vidare till n�sta axiom.

\begin{axiom}
	\label{ax:NaturligaAxiomEfterfoljare}
	F�r alla naturliga tal \(a\) existerar en efterf�ljare betecknad \(S(a)\)
	som �r ett naturligt tal.
\end{axiom}
F�r ett naturligt tal \(n\), l�ter vi dess efterf�ljare betecknas med \(S(n)\).
P� samma s�tt l�ter vi \(S(S(n))\) beteckna efterf�ljaren till efterf�ljaren
till \(n\), och s� vidare.
Notera att vi vet �nnu inte vad som menas med en efterf�ljare.
Det enda vi vet �r att alla naturliga tal har en efterf�ljare som �r ett
naturligt tal.
Vi kan d�rmed fylla upp v�r matematikv�rld med en l�ng rad efterf�ljare och
efterf�ljare till efterf�ljare och s� vidare.
\begin{axiom}
	\label{ax:NaturligaAxiomEjCirkular}
	F�r alla naturliga tal \(n\) g�ller att \(0\) inte �r dess efterf�ljare.
\end{axiom}
Vi kommer att �terkomma till de tv� senaste axiomen.
Men l�t oss f�rst g� vidare med vad vi menar med likhet och att tv� naturliga
tal �r lika.
Vi betecknar likhet med tecknet \(=\).
\begin{axiom}[Reflexivitet]
	\label{ax:NaturligaAxiomReflexivitet}
	F�r alla naturliga tal \(n\) g�ller att \(n\) �r lika med sig sj�lv.
	Detta betecknas \(n=n\).
\end{axiom}
Vi kan nu ocks� konstatera i v�r matematikv�rld att \(0=0\), och vi vet att
\(S(0)=S(0), S(S(0)) = S(S(0)), \ldots, S(S(\ldots S(0))) = S(S(\ldots
S(0)))\).
\begin{axiom}[Symmetri]
	\label{ax:NaturligaAxiomSymmetri}
	F�r alla naturliga tal \(a\) och \(b\) g�ller att om \(a\) �r lika med
	\(b\) d� �r �ven \(b\) lika med \(a\).
	Det vill s�ga, om \(a=b\) d� �r \(b=a\).
\end{axiom}
\begin{axiom}[Transitivitet]
	\label{ax:NaturligaAxiomTransitivitet}
	F�r alla naturliga tal \(a,b\) och \(c\) g�ller att
	om \(a=b\) och \(b=c\) d� �r \(a=c\).
\end{axiom}
\begin{axiom}[Slutenhet under likhet]
	\label{ax:NaturligaAxiomSlutenhet}
	F�r alla naturliga tal \(a\) g�ller att om \(a=b\) f�r n�got \(b\) d� m�ste
	\(b\) ocks� vara ett naturligt tal.
\end{axiom}
\axref{ax:NaturligaAxiomReflexivitet}, \ref{ax:NaturligaAxiomSymmetri},
\ref{ax:NaturligaAxiomTransitivitet} och \ref{ax:NaturligaAxiomSlutenhet}
behandlar begreppet likhet (\(=\)).
\axref{ax:NaturligaAxiomReflexivitet} s�ger att ett naturligt tal m�ste
vara lika med sig sj�lvt.
Denna egenskap kallas reflexivitet\index{reflexivitet}.
\axref{ax:NaturligaAxiomSymmetri} s�ger att om ett naturligt tal �r lika
med ett annat, d� m�ste �ven omv�ndningen g�lla.
Denna egenskap kallas symmetri\index{symmetri}.
\axref{ax:NaturligaAxiomTransitivitet} s�ger att om vi f�r en kedja med
likheter, d� m�ste �ndarna av kedjorna vara lika.
Exempelvis, om \(a=b\) och \(b=c\) f�r vi att \(a=b=c\) och \(a=c\) m�ste d�
g�lla.
Denna egenskap kallas transitivitet\index{transitivitet}.
\axref{ax:NaturligaAxiomSlutenhet} s�ger att om ett naturligt tal �r lika
med n�gonting, d� m�ste detta n�gonting ocks� vara ett naturligt tal.
Denna egenskap kallas slutenhet\index{slutenhet}, hur vi �n anv�nder likhet kan
vi inte komma utanf�r de naturliga talen.
Vi vet nu hur begreppet likhet och \(=\) ska fungera och vad det betyder.

Vi ska nu g� tillbaka till \axref{ax:NaturligaAxiomEfterfoljare} och
\axref{ax:NaturligaAxiomEjCirkular}.
Vi ska dock f�rst introducera ytterligare ett axiom som vi vill kombinera med
dessa tv� axiom.
\begin{axiom}
	\label{ax:NaturligaAxiomInjektion}
	F�r alla naturliga tal \(a\) och \(b\) g�ller att om deras efterf�ljare �r
	lika m�ste �ven \(a\) och \(b\) vara lika.
	Det vill s�ga, om \(S(a)=S(b)\) d� �r \(a=b\).
\end{axiom}
\axref{ax:NaturligaAxiomInjektion} s�ger att tv� olika tal kan inte ha samma
efterf�ljare.
Detta betyder att vi inte kan f� exempelvis grenstruktur eller ''�glor''.
Utan strukturen som m�ste uppst� �r en linje d�r varje naturligt tal �r en
efterf�ljare till ett unikt annat naturligt tal -- med undantag f�r noll
(\(0\)) som enligt \axref{ax:NaturligaAxiomEjCirkular} inte �r efterf�ljare till
n�got naturligt tal.
\axref{ax:NaturligaAxiomEfterfoljare} s�ger att ett naturligt tals
efterf�ljare alltid �r ett naturligt tal och att en s�dan alltid existerar.
Dessa tv� axiom s�ger tillsammans med \axref{ax:NaturligaAxiomInjektion} att
det finns o�ndligt m�nga naturliga tal.
Om vi har ett naturligt tal kan vi alltid ta dess efterf�ljare enligt
\axref{ax:NaturligaAxiomEfterfoljare}, men oavsett hur m�nga efterf�ljare vi
tar kommer vi enligt \axref{ax:NaturligaAxiomEjCirkular} aldrig tillbaka dit vi
startade vid \(0\).
Vi vet att inget naturligt tal kan ha \(0\) som efterf�ljare, men \(S(0)\) d�?
Om vi l�ter \(S(S(0))\) ha \(S(0)\) som efterf�ljare, d� f�r vi en �gla trots
att det inte har \(0\) som efterf�ljare.
D�rf�r beh�ver vi \axref{ax:NaturligaAxiomInjektion} som s�ger att d� m�ste
\(S(S(0))\) och \(0\) vara samma naturliga tal -- vilket inte �r sant och
f�ljdaktligen kan vi inte f� n�gra �glor.

Vi tittar nu p� det sista axiomet.
\begin{axiom}[Induktionsaxiomet]
	\label{ax:NaturligaAxiomInduktion}
	L�t \(M\) vara en samling av objekt s�dan att \(0\) tillh�r \(M\) och har
	egenskapen att det f�r alla naturliga tal \(n\) g�ller att om \(n\) tillh�r
	samlingen \(M\) d� tillh�r �ven efterf�ljaren \(S(n)\) samlingen \(M\).
	D� inneh�ller \(M\) alla naturliga tal.
\end{axiom}
Det sista axiomet, \axref{ax:NaturligaAxiomInduktion}, beskriver
induktionsprincipen, de naturliga talen i sig och �ven m�ngden av alla
naturliga tal.
Det s�ger att om \(0\) tillh�r en samling och efterf�ljaren till varje
naturligt tal i samlingen finns med, d� inneh�ller m�ngden alla
naturliga tal.
Noll (\(0\)) �r ett naturligt tal, d� finns efterf�ljaren \(S(0)\) ocks� med.
Eftersom att efterf�ljaren \(S(0)\) till \(0\) �r ett naturligt tal, d� m�ste
�ven \(S(S(0))\) vara med i denna samling.
D� s�ger vi att samlingen m�ste inneh�lla alla naturliga tal.
Det f�ljer ocks� fr�n detta axiom att alla naturliga tal �r p� formen
\(S(S(\ldots S(0)))\).
Det �r detta axiom som ligger till grund f�r bevismetoden induktion, d�rav
axiomets namn.
%\begin{exercise}
%	Genom diskussion j�mf�r induktionsaxiomet,
%	\axref{ax:NaturligaAxiomInduktion}, med hur induktionsbevis fr�n
%	\chref{ch:Logik} genomf�rs.
%\end{exercise}
% TODO introducera och bevisa induktionsbevis.

Vi ska nu inf�ra n�gra v�lbekanta symboler.
\begin{definition}
	\label{def:NaturligaBeteckningar}
	L�t f�ljande symboler beteckna de olika efterf�ljarna.
	\begin{equation*}
		1 = S(0),\quad
		2 = S(1),\quad
		3 = S(2),\quad
		\ldots
	\end{equation*}
	L�t dessutom \(\N=\{0,1,2,3,\ldots\}\) beteckna m�ngden av alla naturliga
	tal.
\end{definition}
\begin{remark}
	M�rk v�l att \(1,2\) och \(3\) enbart utg�r symboler f�r
	\[
	S(0),S(S(0))\text{ respektive }S(S(S(0))).
	\]
	Vi vet inte hur dessa f�rh�ller sig till varandra genom addition och
	multiplikation eftersom vi inte vet vad addition och multiplikation �r
	�nnu.
	Vi vet inte ens om dessa objekt som symbolerna representerar g�r att r�kna
	med �nnu.
	�nnu utg�r dessa bara en oordnad ansamling av symboler.
\end{remark}

% TODO fixa definition av S^n(0) = S(S(...S(0)))
Vi ska nu g�ra en definition som kommer att f�renkla v�r notation avsev�rt.
%Denna definition anv�nder faktiskt induktionsprincipen fr�n induktionsaxiomet.
%\begin{definition}
%	\label{def:Efterfoljarpotenser}
%	L�t \(S^0(0) = 0\).
%	Om \(S^n(0)\) �r definierad f�r n�got naturligt tal \(n\), d� l�ter vi
%	\(S^{S(n)}(0) = S(S^n(0))\).
%\end{definition}
%\begin{example}
%	Vi ska titta n�rmare p� det naturliga talet \(3=S^3(0)\).
%	Vi har att \(0=S^0(0)\) och \(1=S(0)\), men d� m�ste \(S(0)=S(S^0(0))\).
%	Enligt \defref{def:Efterfoljarpotenser} �r detta samma sak som
%	\(S^{S(0)}(0)=S^1(0)\).
%	Vi har ocks� att \(2=S(S(0))\).
%	P� samma s�tt som tidigare f�r vi att \(S(S(0))=S(S^1(0))\) och vi kan nu
%	forts�tta med att konstatera att detta �r lika med \(S^{S(1)}(0)=S^2(0)\).
%	Vi kan nu anv�nda detta resultat f�r att se att \(3=S(S(S(0)))\) faktiskt
%	�r \(S^3(0)\).
%\end{example}
\begin{definition}
	Om \(n\) �r ett naturligt tal och \(n=S(S(\cdots S(0)))\).
	D� skriver vi \(S^n(0) = n\).
	Om \(n=0\) skriver vi \(S^0(0) = 0\).
\end{definition}



%%%%%%%%%%%%%%%%%%%%%%%%%%%%%%%%%%%%%
% VON NEUMANNS KONSTRUKTION
%%%%%%%%%%%%%%%%%%%%%%%%%%%%%%%%%%%%%
%\section[von Neumanns konstruktion av de naturliga talen]
%	{von Neumanns konstruktion av de naturliga talen\maybeskip}
%\label{sec:vonNeumannNaturliga}
%\noindent
%... (i m�n av tid.)


%%%%%%%%%%%%%%%%%%%%%%%%%%%%%%%%%%%%%
% INDUKTIONSBEVIS
%%%%%%%%%%%%%%%%%%%%%%%%%%%%%%%%%%%%%
%\section[Induktionsbevis]{Induktionsbevis\maybeskip}
%\noindent
%\lettrine{D}{et �r nu} dags att bevisa induktionsprincipen f�r bevis som vi tog
%upp i \chref{ch:Logik}.
%...


%%%%%%%%%%%%%%%%%%%%%%%%%%%%%%%%%%%%%
% ARITMETIK
%%%%%%%%%%%%%%%%%%%%%%%%%%%%%%%%%%%%%
\section[Aritmetik]{Aritmetik\maybeskip}
\index{aritmetik}
\noindent
\lettrine{O}{rdet aritmetik kommer} fr�n grekiskans \emph{\ibygr{a)riqmo)s}},
som betyder tal, och \emph{\ibygr{a)riqmhtikh)}}, som betyder konsten att
r�kna\nocite{OED}.
Aritmetiken kan beskrivas som l�ran om att kombinera tal.
De delar av aritmetiken vi ska behandla i detta avsnitt �r operationerna
addition (\(+\)) och multiplikation (\(\cdot\)).
Det vill s�ga, vi ska i detta avsnitt best�mma hur man r�knar med de naturliga
talen.

Innan vi g�r vidare till att titta p� addition och multiplikation beh�ver vi en
definition.
\begin{definition}
	\index{bin�r operation}
	En \emph{bin�r operation}\index{bin�r operation} \(\diamond\) p� en m�ngd
	\(M\) �r en funktion \(\diamond\colon M\times M\to M\) som tar tv�
	element \(x\) och \(y\) i \(M\) och parar dessa med ett element
	\(\diamond(x,y)\) i \(M\).
	Vanligtvis betecknas \(\diamond(x,y)\) med \(x\diamond y\).
	Det vill s�ga, \((x,y)\mapsto x\diamond y\).
\end{definition}


%%%%%%%%%%%%%%%%%%%%%%%%%%%%%%%%%%%%%
% ADDITION
%%%%%%%%%%%%%%%%%%%%%%%%%%%%%%%%%%%%%
\subsection{Addition}
%\section{Addition}
\noindent
%\lettrine{D}{en f�rsta} ...
Den f�rsta av de aritmetiska operationerna vi ska ta upp �r addition.
Den definition vi anv�nder oss av i detta kompendium �r samma definition som
gavs av Peano �r 1889.
Peanos definition av addition bygger �ven den p� induktionsprincipen och kan
d�rf�r till en b�rjan kanske upplevas lite underlig och sv�rf�rst�elig, men vi
ska diskutera den efter�t.
\begin{definition}[Summa]
	\label{def:NaturligaSumma}
	\index{summa}\index{term}
	\index{\(+\)}
	F�r varje par av naturliga tal \(a\) och \(b\) existerar en unik
	\emph{summa} \(a+b\) som �r ett naturligt tal.
	Delarna \(a\) och \(b\) av en summa kallas f�r summans \emph{termer}.
	Vi definierar f�rst
	\begin{equation}
		\label{eq:AdditionNoll}
		a+0 = a.
	\end{equation}
	Om summan \(a+b\) �r definierad l�ter vi
	\begin{equation}
		\label{eq:AdditionRekursion}
		a+S(b) = S(a+b).
	\end{equation}
\end{definition}
Den f�rsta delen av definitionen �r t�mligen enkel.
Allt \eqref{eq:AdditionNoll} s�ger �r att om vi adderar noll fr�n h�ger till
ett tal s� f�r vi talet sj�lvt.
Det vill s�ga, det h�nder ingenting vid addition med noll fr�n h�ger.
Detta �r dock v�ldigt viktigt, och vi kommer att se varf�r alldeles strax. 

Den andra delen kan upplevas lite sv�rare.
Det \eqref{eq:AdditionRekursion} s�ger �r att ett tal adderat med efterf�ljaren
till ett annat �r samma sak som efterf�ljaren till de b�da talens summa.
Men hur hj�lper det oss?
Det visas l�ttast med ett exempel.
\begin{example}
	Vi vill finna summan f�r talen \(2\) och \(3\).
	Alla tal kan skrivas som en kedja av efterf�ljare till noll, vi vet fr�n
	\defref{def:NaturligaBeteckningar} att \(2=S(S(0))\) och \(3=S(S(S(0)))\).
	Om vi skriver summan \(2+3\) p� formen fr�n
	\eqref{eq:AdditionRekursion} har vi \(2+S(S(S(0)))=S(2+S(S(0)))\).
	Men d� fick vi ett nytt uttryck \(2+S(S(0))\) som �r p� samma form, summan
	av ett tal och efterf�ljaren till ett tal.
	Om vi anv�nder \eqref{eq:AdditionRekursion} igen f�r vi
	\(S(2+S(S(0)))=S(S(2+S(0)))\).
	Nu har vi �terigen ett uttryck p� samma form.
	Upprepning ger oss \(S(S(2+S(0)))=S(S(S(2+0)))\).
	Nu fick vi dock inte en summa av ett tal och en efterf�ljare, utan
	vi fick summan av ett tal och noll.
	Men vi vet ju fr�n \eqref{eq:AdditionNoll} att \(2+0=2\) och d� f�r vi
	\(S(S(S(2)))\).
	Det �r just detta som g�r \eqref{eq:AdditionNoll} s� viktig, f�rr eller
	senare kommer vi fram till en summa d�r ena termen �r noll och d� m�ste vi
	veta vad det �r.
	Ut�ver detta vi vet ocks� att \(2=S(S(0))\), om vi s�tter in detta f�r vi
	\(S(S(S(S(S(0)))))\) som vi enligt definition betecknar med \(5\).
	F�ljdaktligen �r \(2+3=5\).
\end{example}

Denna typ av �terupprepande anv�ndning av sig sj�lv kallas f�r
\emph{rekursion}\index{rekursion}.

\begin{exercise}
	Visa att om \(a\) �r ett naturligt tal, d� �r \(a+1=S(a)\).
\end{exercise}
\begin{exercise}
	Visa att om \(a\) och \(b\) �r naturliga tal, d� �r \(a+b=S^b(a)\).
\end{exercise}

Notera att \(a+0\) per definition �r lika med \(a\), detta s�ger tyv�rr
ingenting om \(0+a\).
\begin{exercise}
	�r \(0+a\) ocks� lika med \(a\)?
	Bevisa ditt p�st�ende.
\end{exercise}

Om vi studerar summan ser vi att \(+\) �r en bin�r operation p� m�ngden av
naturliga tal.
Vi kallar denna operation f�r \emph{addition}.
\begin{definition}[Addition]
	\label{def:NaturligaAddition}
	\index{addition}\index{naturliga tal!addition}
	Additionsoperatorn \(+\) �r en funktion \(+\colon\N\times\N\to\N\) s�dan att
	varje par av naturliga tal \((a,b)\) avbildas p� summan \(a+b\), som �r
	ett naturligt tal som vi finner genom \defref{def:NaturligaSumma}.
\end{definition}



%%%%%%%%%%%%%%%%%%%%%%%%%%%%%%%%%%%%%
% IDENTITETSELEMENT
%%%%%%%%%%%%%%%%%%%%%%%%%%%%%%%%%%%%%
\subsection{Identitetselementet}
\noindent
N�r vi nu sett nollans speciella betydelse �r det dags att ge dess viktiga
egenskap ett namn.
Detta g�r vi i f�ljande definition.
\begin{definition}
	\index{identitetselement}%\index{enhet}
	Givet en m�ngd \(M\) med en definierad bin�r operation \(\diamond:M\times
	M\to M\).
	Ett element \(e\) kallas \emph{identitetselement} om det f�r alla element
	\(x\) i m�ngden uppfyller att \(x\diamond e=e\diamond x=x\).
\end{definition}
\begin{example}
	Det naturliga talet \(0\) �r det additiva identitetselementet f�r de
	naturliga talen.
\end{example}
Man kan nu undra varf�r vi vill ha en s�dan definition enbart f�r nollan?
Anledningen �r att det finns andra tal bland de naturliga talen som beter
precis som nollan, fast f�r en annan operation �n addition.
Vi kommer att st�ta p� ett identitetselement till i n�sta avsnitt.


%%%%%%%%%%%%%%%%%%%%%%%%%%%%%%%%%%%%%
% MULTIPLIKATION
%%%%%%%%%%%%%%%%%%%%%%%%%%%%%%%%%%%%%
\subsection{Multiplikation}
\noindent
Multiplikation �r den andra aritmetiska operationen vi ska titta p� i detta
kapitel.
�ven definitionen av multiplikation �r den Peano gav �r 1889.
Dessutom bygger �ven den p� rekursion, precis som definitionen f�r addition.
\begin{definition}[Produkt]
	\label{def:NaturligaProdukt}
	\index{produkt}\index{faktor}
	\index{\(\cdot\)}
	F�r varje par av naturliga tal \(a\) och \(b\) existerar en unik
	\emph{produkt} \(a\cdot b\) som �r ett naturligt tal.
	%Det vill s�ga \((a,b)\in\N\times\N\) och \(a\cdot b\in\N\), och allts�
	%\(\cdot :\N\times\N\to\N\) d�r \((a,b)\mapsto a\cdot b\).
	Delarna \(a\) och \(b\) av en produkt kallas f�r produktens \emph{faktorer}.
	Vi definierar f�rst
	\begin{equation}
		a\cdot 0 = 0.
	\end{equation}
	Om produkten \(a\cdot b\) �r definierad l�ter vi
	\begin{equation}
		a\cdot S(b) = a+(a\cdot b).
	\end{equation}
	Produkten \(a\cdot b\) skrivs vanligen som \(ab\).
\end{definition}

Likt summan ger �ven produkten en bin�r operation.
\begin{definition}[Multiplikation]
	\label{def:NaturligaMultiplikation}
	\index{multiplikation}\index{naturliga tal!multiplikation}
	Multiplikationsoperatorn \(\cdot\) �r en funktion
	\(\cdot\colon\N\times\N\to\N\) s�dan att varje par av naturliga tal
	\((a,b)\) avbildas p� produkten \(a\cdot b\), som �r ett naturligt tal som
	vi finner genom \defref{def:NaturligaProdukt}.
	Denna operation kallar vi f�r \emph{multiplikation}.
\end{definition}

\begin{exercise}
	Vilket element �r identitetselementet f�r multiplikation av de naturliga
	talen?
	Visa att s� �r fallet.
\end{exercise}
\begin{exercise}
	Visa att om \(a\) och \(b\) �r naturliga tal, d� �r \[a\cdot b =
	\underbrace{a+a+\cdots+a}_b.\]
\end{exercise}
\begin{exercise}
	Visa att \(0\cdot a = 0\).
	Notera att \(a\cdot 0\) per definition �r lika med \(0\), \(0\cdot a = 0\)
	fordrar dock ett bevis.
\end{exercise}


%%%%%%%%%%%%%%%%%%%%%%%%%%%%%%%%%%%%%
% LIKHET OCH OLIKHET
%%%%%%%%%%%%%%%%%%%%%%%%%%%%%%%%%%%%%
\section{Likhet och olikhet}
\noindent
\lettrine{D}{et �r nu dags} att introducera ett s�tt att j�mf�ra tal som ej �r
lika.
Vi har redan sett likhet, som vi betecknade med \(=\) och utl�ste \emph{�r lika
med}.
Likheter �r v�ldigt intressanta, men det finns m�nga saker som inte �r lika.
Exempelvis har vi de naturliga talen, de �r o�ndligt m�nga och inget av dem �r
lika med n�got annat.
D�rf�r definierar vi h�r en annan relation p� de naturliga talen.

\begin{definition}[Olikhet]
	\index{olikhet}\index{naturliga tal!olikhet}
	\index{<}\index{\(\leq\)}\index{>}\index{\(\geq\)}
	L�t \(a\) och \(b\) vara naturliga tal.
	D� s�ger vi att \(a\) �r \emph{mindre �n eller lika med} \(b\) om det finns
	ett naturligt tal \(n\) s�dant att \(a+n=b\), vi skriver detta som
	\(a\leq b\).
	Vi kan ocks� s�ga att \(b\) �r \emph{st�rre �n eller lika med} \(a\) och
	beteckna detta genom \(b\geq a\).
	Om vi ej till�ter \(n\) att vara noll, d� skriver vi \(a<b\)
	respektive \(b>a\).
	Vi utl�ser dessa som \(a\) �r \emph{strikt mindre �n} \(b\)
	respektive \(b\) �r \emph{strikt st�rre �n} \(a\).
\end{definition}

\begin{example}
	Vi kan nu s�ga att \(0<1<2<3\) och s� vidare.
	Det vill s�ga, vi har nu inf�rt en form av ordning av de naturliga talen.
\end{example}

\begin{example}
	L�t \(x\) vara ett naturligt tal s�dant att \(0<x\) och \(x<5\), det vill
	s�ga \(0<x<5\).
	Vi menar d� att \(x\) kan vara n�got av talen \(1,2,3\) eller \(4\).
\end{example}

\begin{exercise}
	�r \(<\), \(\leq\), \(>\) och \(\geq\) relationer?
\end{exercise}


%%%%%%%%%%%%%%%%%%%%%%%%%%%%%%%%%%%%%
% EGENSKAPER F�R ADDITION
%%%%%%%%%%%%%%%%%%%%%%%%%%%%%%%%%%%%%
\section[Additionens algebraiska egenskaper]
	{Additionens algebraiska egenskaper\maybeskip}
\noindent
\lettrine{O}{rdet algebra kommer} fr�n arabiskans \emph{al-jabr} genom Muhammad
ibn M\={u}s\={a} al-Khw\={a}rizm\={\i}s (ca. 780-ca. 850) bok \emph{Al-Kit\={a}b
al-mukhta\d{s}ar f\={\i} h\={\i}s\={a}b al-\u{g}abr wa'l-muq\={a}bala}, p�
svenska \emph{Den sammanfattande boken om ber�kning genom komplettering och
balansering}\footnote{Egen �vers�ttning fr�n ''The Compendious Book on
Calculation by Completion and Balancing''.}.
Ordet \emph{al-jabr} betyder ordagrant \emph{�terst�llande}.
Algebra kan beskrivas som matematikens studium av operationer och regler.
Vi ska nu titta n�rmare p� hur den aritmetiska operationen addition beter sig.

Innan vi tittar p� additionens algebraiska egenskaper beh�ver vi n�gra
hj�lpsatser.
Vi b�rjar med en mycket enkel hj�lpsats som s�ger att om man adderar talet ett
till ett tal f�r man dess efterf�ljare.
\begin{lemma}\label{lem:NaturligaAdderaEtt}
	Om \(a\) �r ett naturligt tal, d� �r \(a+1\) dess efterf�ljare.
\end{lemma}
\begin{proof}
	Om vi tittar p� additionen \(a+1\) har vi per definition att \(a+1=a+S(0)\).
	Fr�n \eqref{eq:AdditionRekursion} f�r vi att \(a+S(0)=S(a+0)\).
	Vi har fr�n \eqref{eq:AdditionNoll} att \(a+0=a\).
	Vi f�r d� att
	\begin{equation}\label{eq:NaturligaAdderaEtt}
		a+1 = a+S(0) = S(a+0) = S(a).
	\end{equation}
	S�ledes �r \(a+1\) efterf�ljaren \(S(a)\) till \(a\).
\end{proof}
\begin{example}
	Om vi vidare tittar p� additionen \(a+2\) har vi att
	\begin{equation*}
		a+2 = a+S(1) = S(a+1).
	\end{equation*}
	Vi har fr�n \lemref{lem:NaturligaAdderaEtt} att \(a+1=S(a)\) och vi f�r att
	\begin{equation*}
		a+2=S(S(a))=(a+1)+1.
	\end{equation*}
\end{example}
Vi kan nu formulera en vidareutveckling av resultatet i exemplet genom f�ljande
hj�lpsats.
\begin{lemma}
	\label{lem:NaturligaAdderaN}
	Om \(a\) och \(n\) �r naturliga tal g�ller att
	\begin{equation}
		a+n=S^n(a).
	\end{equation}
\end{lemma}
\begin{proof}
	Vi har enligt \defref{def:NaturligaBeteckningar} att \(n=S^n(0)\).
	D� har vi att \(a+n=a+S^n(0)\).
	Enligt \eqref{eq:AdditionRekursion} �r \(a+S^n(0) = S(a+S^{n-1}(0))\), d�r
	\(n-1\) f�r beteckna det naturliga tal s�dant att \(S(n-1)=n\).
	Men enligt \eqref{eq:AdditionRekursion} �r \(a+S^{n-1}(0)=S(a+S^{n-2}(0))\),
	d�r \(n-2\) �r det naturliga tal s�dant att \(S^2(n-2)=n\).
	Vi har nu att \[a+n=S(S(a+S^{n-2}(0)))=S^2(a+S^{n-2}(0)).\]
	S�ledes har vi \(S^k(a+S^{n-k}(0))\) f�r \(k\leq n\).
	N�r \(k=n\) f�r vi \(S^n(a+S^{n-n}(0)) = S^n(a+0) = S^n(a)\).
\end{proof}

\begin{exercise}
	Visa att \(S^a(S^b(0)) = S^{b+a}(0)\).
\end{exercise}

Vi kan nu b�rja titta p� vilka egenskaper som addition har.
En fr�ga som vi kan st�lla oss, spelar det n�gon roll i vilken ordning vi
adderar?
Spelar det n�gon roll om vi adderar f�rst \(1\) och \(2\) och sedan adderar
\(3\)?
F�ljande sats besvarar just den fr�gan.
\begin{theorem}[Associativitet]
	\label{thm:NaturligaAssociativitet}\label{thm:NaturligaAdditionAssociativ}
	\index{associativitet}\index{naturliga tal!associativitet}
	Om \(a,b\) och \(c\) �r naturliga tal, d� g�ller att \(a+(b+c)=(a+b)+c\).
\end{theorem}
\begin{proof}
	Vi b�rjar med att titta p� \(a+(b+c)\).
	Enligt \lemref{lem:NaturligaAdderaN} har vi att \(b+c=S^c(b)\).
	Enligt \defref{def:NaturligaBeteckningar} �r \(b=S^b(0)\) och vi f�r
	att \(b+c=S^c(S^b(0)).\)
	Vi har d� kvar \(a+(b+c)=a+S^c(S^b(0))\).
	Enligt \lemref{lem:NaturligaAdderaN} igen har vi
	\begin{equation}\label{eq:NaturligaAssociativ1}
		a+S^c(S^b(0))=S^c(S^b(a))=S^c(S^b(S^a(0))).
	\end{equation}

	Vi forts�tter med att kolla p� \((a+b)+c\).
	D� har vi \(a+b=S^b(a)\) och s�ledes
	\begin{equation}\label{eq:NaturligaAssociativ2}
		S^b(a)+c=S^c(S^b(a))=S^c(S^b(S^a(0))).
	\end{equation}

	Vi ser att \eqref{eq:NaturligaAssociativ1} och
	\eqref{eq:NaturligaAssociativ2} �r lika och s�ledes har vi �ven att
	\[a+(b+c)=(a+b)+c.\]
\end{proof}

\begin{example}
	Vi vill addera talen \(1,2\) och \(3\) genom \(1+2+3\).
	Enligt \thmref{thm:NaturligaAssociativitet} spelar det ingen roll om vi
	f�rst adderar \(1+2=3\) och sedan adderar \(3\), det vill s�ga \(3+3=6\),
	eller om vi f�rst adderar \(2+3=5\) och sedan adderar \(1+5=6\).
	Som vi ser �r \((1+2)+3=3+3=6\) och \(1+(2+3)=1+5=6\) b�da lika med \(6\).
\end{example}

Vidare kan vi fr�ga oss, spelar det n�gon roll om vi adderar \(1\) med \(2\)
eller om vi adderar \(2\) med \(1\)?
%Denna fr�ga besvaras i f�ljande sats.
\begin{theorem}[Kommutativitet]
	\label{thm:NaturligaKommutativitet}
	\index{kommutativitet}\index{naturliga tal!kommutativitet}
	Om \(a\) och \(b\) �r naturliga tal, d� g�ller att \(a+b=b+a\).
\end{theorem}
\begin{proof}
	Vi har fr�n \lemref{lem:NaturligaAdderaN} att \(a+b=S^b(a)=S^b(S^a(0))\).
	Men
	\begin{equation}
		\label{eq:NaturligaKommutativ1}
		S^b(S^a(0)) =
		\underbrace{S(S(S(\cdots(S}_b(\underbrace{S(S(S(\cdots(S}_a(0)))))))))).
	\end{equation}

	P� samma s�tt har vi att \(b+a=S^a(b)=S^a(S^b(0))\) och
	\begin{equation}
		\label{eq:NaturligaKommutativ2}
		S^a(S^b(0)) =
		\underbrace{S(S(S(\cdots(S}_a(\underbrace{S(S(S(\cdots(S}_b(0)))))))))).
	\end{equation}

	Eftersom att \eqref{eq:NaturligaKommutativ1} och
	\eqref{eq:NaturligaKommutativ2} �r lika m�ste vi ha \(a+b=b+a\).
\end{proof}

\begin{example}
	Vi vill addera talen \(1\) och \(2\).
	Enligt \thmref{thm:NaturligaKommutativitet} spelar det ingen roll om vi g�r
	detta genom \(1+2\) eller \(2+1\).
	I b�da fallen kommer vi fram till att \(1+2=2+1=3\).
\end{example}



%%%%%%%%%%%%%%%%%%%%%%%%%%%%%%%%%%%%%
% EGENSKAPER F�R MULTIPLIKATION
%%%%%%%%%%%%%%%%%%%%%%%%%%%%%%%%%%%%%
\section[Multiplikationens algebraiska egenskaper]
	{Multiplikationens algebraiska egenskaper\maybeskip}
\noindent
\lettrine{P}{recis som f�r} addition undrar vi nu hur multiplikation beter sig.
Spelar det n�gon roll i vilken ordning vi multiplicerar naturliga tal?
En ytterligare fr�ga som uppst�r nu �r dock: hur f�rh�ller sig multiplikation
till addition?
Vi ska b�rja med att besvara denna fr�ga, sedan forts�tter vi med att unders�ka
associativiteten och kommutativiteten som vi gjorde f�r addition.
\begin{theorem}[Distributivitet]
	\label{thm:NaturligaDistributivitet}
	\index{distributivitet}\index{naturliga tal!distributivitet}
	Om \(a,b\) och \(c\) �r naturliga tal g�ller att
	\(a\cdot(b+c) = a\cdot b + a\cdot c\) och \((a+b)\cdot c = a\cdot c +
	b\cdot c\).
\end{theorem}
\begin{proof}
	Vi har f�rst fr�n \lemref{lem:NaturligaAdderaN} att \(b+c=S^c(S^b(0))\).
	S�ledes f�r vi att \(a\cdot (b+c) = a\cdot (S^c(S^b(0)))\).
	Fr�n \defref{def:NaturligaMultiplikation} f�r vi d� att
	\begin{multline}
		\label{eq:NaturligaDistributivitet}
		a\cdot (b+c) = a\cdot (S^c(S^b(0))) = \underbrace{a+a+\cdots+a}_{c} +
			a\cdot S^b(0) =
			\underbrace{a+a+\cdots+a}_{c}+\underbrace{a+a+\cdots+a}_{b}.
	\end{multline}
	Om vi tittar p� de enskilda delarna av uttrycket l�ngst till h�ger.
	D� har vi enligt definitionen f�r produkten att
	\begin{align*}
		\underbrace{a+a+\cdots+a}_{b}
			&= \underbrace{a+a+\cdots+a}_{b-1}+a\cdot S(0) \\
			&= \underbrace{a+a+\cdots+a}_{b-2}+a\cdot S(1) \\
			&= \underbrace{a+a+\cdots+a}_{b-3}+a\cdot S(2) \\
			& \vdots \\
			&= a\cdot b.
	\end{align*}
	Vi har p� samma s�tt att
	\begin{equation*}
		\underbrace{a+a+\cdots+a}_{c} = a\cdot c.
	\end{equation*}
	Om vi anv�nder detta i \eqref{eq:NaturligaDistributivitet} f�r vi att
	\begin{equation*}
		a\cdot (b+c) = \underbrace{a+a+\cdots+a}_b+\underbrace{a+a+\cdots+a}_c
			= a\cdot b + a\cdot c,
	\end{equation*}
	vilket visar f�rsta delen av satsen.

	Om vi tittar p� \((a+b)\cdot c\) f�r vi att
	\begin{equation*}
		(a+b)\cdot c = \underbrace{(a+b)+(a+b)+\cdots+(a+b)}_c.
	\end{equation*}
	Eftersom att additionen �r associativ och kommutativ kan vi byta plats
	p� termerna och f�r d�
	\begin{equation*}
		\underbrace{(a+b)+(a+b)+\cdots+(a+b)}_c =
		\underbrace{a+a+\cdots+a}_c+\underbrace{b+b+\cdots+b}_c.
	\end{equation*}
	P� samma s�tt som ovan f�r vi att detta �r \(a\cdot c + b\cdot c\).
	D� har vi visat att \((a+b)\cdot c = a\cdot c + b\cdot c\) och vi har visat
	satsen.
\end{proof}

\begin{exercise}
	G�ller det ocks� att
	\(a\cdot (b_1+b_2+\cdots+b_n) =
		a\cdot b_1 + a\cdot b_2 + \cdots + a\cdot b_n\)?
\end{exercise}

Vi vet nu hur multiplikationen f�rh�ller sig till additionen och kan d� g�
vidare till att unders�ka om multiplikationen har de associativa och
kommutativa egenskaperna som additionen har.
Vi b�rjar med associativiteten.
\begin{theorem}[Associativitet]
	\label{thm:NaturligaMultiplikationAssociativ}
	\index{associativitet}\index{naturliga tal!associativitet}
	Om \(a\) och \(b\) �r naturliga tal, d� g�ller att \(a\cdot(b\cdot
	c)=(a\cdot b)\cdot c\).
\end{theorem}
\begin{proof}
	Vi tittar f�rst p� \(a\cdot (b\cdot c)\) och f�r enligt definitionen f�r
	multiplikation att
	\begin{equation*}
		a\cdot (b\cdot c) = a\cdot (\underbrace{b+b+\cdots+b}_c).
	\end{equation*}
	Eftersom att multiplikationen �r distributiv
	(\thmref{thm:NaturligaDistributivitet}) f�r vi att
	\begin{equation}
		\label{eq:NaturligaMultAssoc1}
		a\cdot (\underbrace{b+b+\cdots +b}_c) = \underbrace{a\cdot b+a\cdot
			b+\cdots+a\cdot b}_c.
	\end{equation}

	Vi tittar nu p� \((a\cdot b)\cdot c\).
	Enligt definitionen f�r multiplikation �r
	\begin{equation}
		\label{eq:NaturligaMultAssoc2}
		(a\cdot b)\cdot c = \underbrace{a\cdot b+a\cdot b+\cdots+a\cdot b}_c.
	\end{equation}

	Eftersom att \eqref{eq:NaturligaMultAssoc1} och
	\eqref{eq:NaturligaMultAssoc2} �r lika har vi visat satsen.
\end{proof}

\begin{theorem}[Kommutativitet]
	\index{kommutativitet}\index{naturliga tal!kommutativitet}
	Om \(a\) och \(b\) �r naturliga tal, d� g�ller att \(a\cdot b=b\cdot a\).
\end{theorem}
\begin{proof}
	Beviset anv�nder induktion.
	L�t \(a\) vara ett naturligt tal.
	Vi vill visa att \(a\cdot b = b\cdot a\) f�r alla naturliga tal \(b\).
	F�r \(b=0\) �r det klart att multiplikationen �r kommutativ.
	F�r \(b=1\) har vi att
	\begin{equation}
		\label{eq:MultKommutativ1}
		a\cdot 1 = a+a\cdot 0 = a.
	\end{equation}
	Vi har ocks� att
	\begin{equation*}
		1\cdot a = \underbrace{1+1+\cdots+1}_a.
	\end{equation*}
	Enligt \lemref{lem:NaturligaAdderaEtt} och att additionen �r associativ
	(\thmref{thm:NaturligaAdditionAssociativ}) f�r vi att
	\begin{equation}
		\label{eq:MultKommutativ2}
		\underbrace{1+1+\cdots+1}_a = S^a(0) = a.
	\end{equation}
	Eftersom att \eqref{eq:MultKommutativ1} och \eqref{eq:MultKommutativ2} �r
	lika m�ste ocks� \(a\cdot 1 = 1\cdot a\).

	Antag att multiplikationen �r kommutativ f�r alla \(b\) mindre �n \(k\).
	Vi har d� att \(a\cdot k = k\cdot a\).
	Vi vill nu visa att d� m�ste multiplikationen vara kommutativ �ven f�r
	\(b=k+1\).
	Eftersom att multiplikation �r distributiv �ver addition
	(\thmref{thm:NaturligaDistributivitet}) har vi att \(a\cdot (k+1) = a\cdot
	k + a\cdot 1\).
	Vi har redan konstaterat att \(a\cdot k = k\cdot a\) och att \(a\cdot 1 =
	1\cdot a\), och f�ljdaktligen �r \(a\cdot (k+1) = a\cdot k + a\cdot 1 =
	k\cdot a + 1\cdot a\).
	Vi har fr�n distributiviteten igen att \(k\cdot a + 1\cdot a = (k+1)\cdot
	a\).
	S�ledes �r multiplikationen kommutativ �ven f�r \(b=k+1\) och den m�ste
	d�rf�r vara kommutativ f�r alla naturliga tal.
\end{proof}
\begin{exercise}
	Vilka likheter finns mellan beviset ovan och induktionsaxiomet f�r de
	naturliga talen?
\end{exercise}

\begin{exercise}
	Visa att \((a+b)(c+d) = ac+ad+bc+bd\).
\end{exercise}


%%%%%%%%%%%%%%%%%%%%%%%%%%%%%%%%%%%
% ALGEBRAISKA EGENSKAPER
%%%%%%%%%%%%%%%%%%%%%%%%%%%%%%%%%%%
\section{Algebraiska egenskaper f�r de naturliga talen}
\label{sec:HeltalensAlgebraiskaEgenskaper}
\noindent
\lettrine{D}{et �r nu dags} att sammanfatta de algebraiska egenskaperna f�r de
naturliga talen.
Vi har i tidigare avsnitt visat att de naturliga talen har f�ljande egenskaper.

\theoremstyle{plain}
\newtheorem*{AlgebraicPropertiesNatural}{Algebraiska egenskaper f�r de
	naturliga talen}
\begin{AlgebraicPropertiesNatural}
	\nocite{Bartle2000itr,Grillet2007aa}
	\label{def:HeltalenEgenskaper}
	P� m�ngden \(\N\) av hela tal definieras tv� bin�ra operationer,
	addition (\(+\)) och multiplikation (\(\cdot\)).
	F�r addition g�ller f�ljande:
	\begin{description}
		\item[Kommutativitet] \(a+b=b+a\) f�r alla \(a,b\in\N\).
		\item[Associtivitet] \((a+b)+c=a+(b+c)\) f�r alla \(a,b,c\in\N\).
		\item[Additivt identitetselement] Det finns ett element \(0\in\N\)
			s�dant att f�r alla \(a\in\N\) g�ller att \(0+a = a+0 = a\).
	\end{description}
	F�r multiplikation g�ller f�ljande:
	\begin{description}
		\item[Kommutativitet] \(a \cdot b=b \cdot a\) f�r alla \(a,b\in\N\).
		\item[Associtivitet] \((a \cdot b) \cdot c=a \cdot (b \cdot c)\) f�r
			alla \(a,b,c\in\N\).
		\item[Multiplikativt identitetselement] Det finns ett element
			\(1\in\N\) s�dant att f�r alla \(a\in\N\) g�ller att
			\(1 \cdot a = a \cdot 1 = a\).
	\end{description}
	Ut�ver detta g�ller �ven
	\begin{description}
		\item[Multiplikativ distributivitet �ver addition]
			\(a \cdot (b+c) = (a \cdot b) + (a \cdot c)\) och
			\((b+c) \cdot a = (b \cdot a) + (c \cdot a)\) f�r alla reella tal
			\(a,b,c\in\N\).
	\end{description}
\end{AlgebraicPropertiesNatural}


%%%%%%%%%%%%%%%%%%%%%%%%%%%%%%%%%%%%%
% POTENSER
%%%%%%%%%%%%%%%%%%%%%%%%%%%%%%%%%%%%%
\section{Potenser}
\noindent
\lettrine{D}{et kan bli} tr�ttsamt i l�ngden att skriva ut m�nga faktorer i en
produkt.
F�r additionen kunde vi ist�llet f�r att skriva \(a+a+\cdots+a\)
multiplicera \(a\) med antalet \(a\)-termer i summan, p� s� vis beh�ver vi inte
skriva ut alla \(a\)-termer utan det r�cker med att vi vet vilken term och hur
m�nga vi ska addera.
Vi vill naturligtvis kunna g�ra liknande f�r multiplikation.
Ist�llet f�r att skriva ut alla \(b\)-faktorer i en produkt \(b\cdot
b\cdots b\) r�cker det med att vi skriver faktorn och antalet av denna
faktor.
D� skriver vi produkten \(b\cdot b\cdots b\) med \(n\) faktorer som \(a^n\).
Detta �stadkommer vi med potenser som definieras enligt f�ljande.
\begin{definition}
	L�t \(a\) och \(n\) vara naturliga tal.
	L�t \(a^1=a^{S(0)}=a\).
	Om \(a^n\) �r definierad l�ter vi \(a^{S(n)}=a\cdot a^n\).
	Vi kallar \(a^n\) f�r en \emph{\(a\)-potens} med \emph{exponenten} \(n\).
\end{definition}
\begin{remark}
	Notera att \(a^0\) ej �r definierad f�r n�got naturligt tal \(a\).
	Vi �terkommer till detta i \chref{ch:Rationella} som handlar om rationella
	tal.
\end{remark}

\begin{example}
	Vi har produkten \(2\cdot 2\cdot 2\) som best�r av tre termer som alla �r
	\(2\).
	Vi kan d� skriva produkten som en \(2\)-potens, n�mligen \(2^3\).
	Enligt definitionen �r \(2^3=2\cdot 2^2=2\cdot 2\cdot 2^1 = 2\cdot 2\cdot
	2\).
\end{example}

\begin{example}
	Talet \(72\) kan skrivas som produkten \(8\cdot 9=2\cdot 2\cdot 2\cdot
	3\cdot 3\).
	Om vi anv�nder potensform f�r vi att \(72=2^3\cdot 3^2=2^3 3^2\).
\end{example}

\begin{example}
	Talet \(4\) kan skrivas som en \(4\)-potens, n�mligen \(4^1\).
	Det kan ocks� skrivas som en \(2\)-potens, n�mligen \(4=2\cdot 2=2^2\).
\end{example}


%%%%%%%%%%%%%%%%%%%%%%%%%%%%%%%%%%%%%
% RESULTAT OM POTENSER
%%%%%%%%%%%%%%%%%%%%%%%%%%%%%%%%%%%%%
\subsection{N�gra resultat om potenser}
\noindent
Vi ska nu titta p� n�gra enkla resultat som f�ljer av v�r definition av
potenser.
En f�rsta fr�ga kan vara, vad h�nder om vi adderar tv� potenser?
\begin{example}
	\label{ex:AdderaPotenser}
	Vi vill addera potenserna \(a^n\) och \(b^m\) d�r \(a\) och \(b\) �r
	naturliga tal och \(n\) och \(m\) �r naturliga tal skilda fr�n noll.
	Om vi adderar dem f�r vi
	\begin{equation*}
		a^n + b^m = \underbrace{a\cdot a\cdots a}_n + \underbrace{b\cdot
		b\cdots b}_m.
	\end{equation*}
	Vi kan inte komma s�rskilt mycket l�ngre och vi kan konstatera att detta
	var ett f�ga intressant resultat.
\end{example}

Om vi ist�llet testar att multiplicera tv� potenser.
\begin{example}
	\label{ex:MultipliceraPotenser}
	Vi vill multiplicera potenserna \(a^n\) och \(b^m\) d�r \(a\) och \(b\) �r
	naturliga tal och \(n\) och \(m\) �r naturliga tal skilda fr�n noll.
	Vi f�r d� att
	\begin{equation*}
		a^n\cdot b^m = a^nb^m = \underbrace{a\cdot a\cdots a}_n\cdot
		\underbrace{b\cdot b\cdots b}_m.
	\end{equation*}
	Detta verkar mer lovande, vad h�nder om \(a=b\)?

	Om \(a=b\) f�r vi att
	\begin{multline}
		\label{eq:MultipliceraPotenser}
		a^nb^m = a^na^m = \underbrace{a\cdot a\cdots a}_n\cdot\underbrace{a\cdot
		a\cdots a}_m = \\
		\underbrace{a\cdot a\cdots a}_{n}\cdot a^m = a^{S^n(m)} = a^{n+m}.
	\end{multline}
\end{example}

Detta �r ett intressant resultat och vi sammanfattar det som f�ljande sats.
\begin{theorem}
	\label{thm:AdderaExponenter}
	Om \(a\), \(n\neq 0\) och \(m\neq 0\) �r naturliga tal, d� �r \(a^na^m =
	a^{n+m}.\)
\end{theorem}
\begin{proof}
	Satsen f�ljer direkt fr�n \eqref{eq:MultipliceraPotenser}.
\end{proof}

\begin{example}
	Vi vill multiplicera potenserna \(2^3\) och \(2^5\).
	Vi f�r d� enligt \thmref{thm:AdderaExponenter} att
	\begin{equation*}
		2^32^5 = 2^{3+5} = 2^8.
	\end{equation*}
	Om vi kontrollerar ser vi att \(2^3=8\) multiplicerat med \(2^5=32\)
	faktiskt �r \(2^8=256\).
\end{example}

Eftersom att det fungerar att addera exponenterna ter det sig naturligt att
fr�ga om vi �ven kan multiplicera exponenterna och vad det skulle betyda.
\begin{example}
	\label{ex:MultipliceraExponenter}
	Om \(a\), \(n\neq 0\) och \(m\neq 0\) �r naturliga tal kan vi skapa
	potensen \(a^{nm}\).
	Vi vet att \[nm = \underbrace{n+n+\cdots+n}_m\] och s�ledes att
	\begin{equation}
		\label{eq:MultipliceraExponenter1}
		a^{nm}=a^{n+n+\cdots+n}=\underbrace{a^na^n\cdots a^n}_m.
	\end{equation}
	Men vi har nyss definierat potenser f�r att f�renkla skrivandet av s�dana
	produkter.
	F�ljdaktligen f�r vi att
	\begin{equation}
		\label{eq:MultipliceraExponenter2}
		a^na^n\cdots a^n = (a^n)^m.
	\end{equation}
\end{example}
Vi sammanfattar resultatet i f�ljande sats.
\begin{theorem}
	L�t \(a\), \(n\neq 0\) och \(m\neq 0\) vara naturliga tal.
	D� g�ller att \((a^n)^m = a^{nm}\).
\end{theorem}
\begin{proof}
	Satsen f�ljer fr�n \exref{ex:MultipliceraExponenter}.
\end{proof}

Vi har nu visat att vi har addition och multiplikation av exponenter.
F�ljande sats visar ocks� att multiplikation av exponenter �r distributiv �ver
addition av exponenter.
\begin{theorem}
	\label{thm:MultipliceraExponenter}
	Om \(a\), \(b\), \(n\neq 0\), \(m\neq 0\) och \(p\neq 0\) �r naturliga tal,
	d� �r \((a^nb^m)^p=a^{np}b^{mp}\).
\end{theorem}
\begin{proof}
	Om vi tittar p� vad \((a^nb^m)^p\) faktiskt betyder, s� finner vi att
	\begin{equation*}
		(a^nb^m)^p = \underbrace{(a^nb^m)(a^nb^m)\cdots (a^nb^m)}_p.
	\end{equation*}
	Eftersom att multiplikation �r associativ och kommutativ f�r vi att
	\begin{equation*}
		\underbrace{(a^nb^m)(a^nb^m)\cdots (a^nb^m)}_p =
		\underbrace{a^na^n\cdots a^n}_p \underbrace{b^mb^m\cdots b^m}_p =
		(a^n)^p(b^m)^p.
	\end{equation*}
	Vi har fr�n \exref{ex:MultipliceraExponenter} att \((a^n)^p(b^m)^p =
	a^{np}b^{mp}.\)
	Och d�rf�r �r \((a^nb^m)^p = a^{np}b^{mp}\).
\end{proof}

Vi avslutar avsnittet med ett exempel som nyttjar de b�da satserna.
\begin{example}
	Vi vill multiplicera \(4^3\) och \(12^2\).
	Vi vet att \(4=2\cdot 2\), det vill s�ga \(4=2^2\), samt att \(12=3\cdot
	4=3\cdot 2^2\).
	Om vi anv�nder detta och tittar p� vad vi hade fr�n b�rjan, \(4^3\) �r
	s�ledes \((2^2)^3=2^{2\cdot 3}=2^6\).
	\(12^2\) blir d� \((3\cdot 2^2)^2\) och s�ledes \(12^2=3^{1\cdot
	2}2^{2\cdot 2}=3^22^4\).
	Om vi multiplicerar dem f�r vi
	\begin{equation*}
		(\underbrace{2^6}_{4^3}\cdot \underbrace{3^22^4}_{12^2}) =
		2^62^43^2 = 2^{6+4}3^2 = 2^{10}3^2.
	\end{equation*}
\end{example}

\begin{exercise}
	Visa att addition av exponenter och multiplikation av exponenter b�da �r
	associativa och kommutativa operationer.
\end{exercise}


%%%%%%%%%%%%%%%%%%%%%%%%%%%%%%%%%%%%%
% V�LORDNINGSPRINCIPEN
%%%%%%%%%%%%%%%%%%%%%%%%%%%%%%%%%%%%%
%\subsection{V�lordningsprincipen}
%\noindent
%...
%
%\begin{theorem}[V�lordningsegenskapen hos de naturliga talen]
%	\label{def:Valordningsprincipen}
%	\index{v�lordningsegenskapen}\index{naturliga tal!v�lordningsegenskapen}
%	Om en m�ngd \(M\subseteq\N\) �r en delm�ngd till de naturliga talen och om
%	\(M\neq\emptyset\) ej �r den tomma m�ngden, d� existerar ett element
%	\(m\in M\) i \(M\) s�dant att \(m\leq k\) �r mindre �n \(k\) f�r alla
%	\(k\in M\) i \(M\).
%\end{theorem}
%\begin{proof}
%	...
%\end{proof}


%%%%%%%%%%%%%%%%%%%%%%%%%%%%%%%%%%%%%
% AVSLUTANDE REFLEKTION
%%%%%%%%%%%%%%%%%%%%%%%%%%%%%%%%%%%%%
\section[Avslutande reflektion]{Avslutande reflektion\maybeskip}
\noindent
\lettrine{S}{om en avslutande} reflektion till detta kapitel ges f�ljande
�vningsuppgifter.
Tanken med dessa �vningar �r att reflektera �ver matematikens existens, hur den
f�rh�ller sig till verkligheten etc.

\begin{exercise}
	Diskutera f�rh�llandet mellan matematiken och verkligheten.
	En inledande fr�ga i diskussionen kan vara:
	Grundar sig matematiken i verkligheten eller �r den oberoende av
	verkligheten?
	Diskutera.
\end{exercise}

\begin{exercise}
	a) Om matematiken �r oberoende av verkligheten, hur kan den anv�ndas
	f�r att utforska och f�ruts�ga verkligheten? Och om verkligheten s�g
	annorlunda ut, skulle matematiken se likadan ut?

	b) Om matematiken grundar sig i verkligheten, hur kan den anv�ndas f�r att
	f�ruts�ga verkligheten n�r den beh�ver verkligheten f�r att visas vara sann?
\end{exercise}

\chapter{De hela talen}
\label{ch:Heltalen}
\noindent
%\lettrine{M}{�ngden av} hela tal $\Z=\{0, -1, 1, -2, 2, \ldots\}$.
\lettrine{V}{i �r nu v�lbekanta} med de naturliga talen,
\(\N=\{0,1,2,\ldots\}\).
Vi har operationerna addition och multiplikation och vet hur dessa fungerar.
I detta kapitel kommer vi att ut�ka de naturliga talen med avseende p�
addition.

Om vi adderar tv� tal \(a\) och \(b\) f�r vi ett tredje tal \(c\), vi skriver
detta som \(a+b=c\).
Det finns dock inget s�tt att ta oss tillbaka till \(a\), det vill s�ga det
finns inget naturligt tal \(d\) s�dant att \(c+d=a+b+d=a\).
Ett annat s�tt att s�ga det p� �r att det finns inga naturliga tal \(n\) och
\(m\) s�dana att deras summa \(n+m=0\) �r lika med identitetselementet noll.
Detta �r dock en v�ldigt intressant och �nskv�rd egenskap.
Vi ska f�rtydliga vad vi menar och ge denna egenskap ett namn.
\begin{definition}
	\label{def:invers}
	\index{invers}
	Givet en m�ngd \(M\) med en definierad bin�r operation \(\diamond:M\times
	M\to M\) och ett identitetselement \(e\).
	Ett element \(b\) kallas f�r \emph{inversen till} ett element \(a\) om
	\(a\diamond b = b\diamond a = e\).
\end{definition}
\begin{exercise}
	Utforska inversbegreppet.
	Fr�gor att inspirera:
	I \defref{def:invers}, �r \(a\) inversen till n�got element?
	Hur m�nga inverser kan ett element ha?
	Har alla element en invers?
\end{exercise}
%\begin{remark}
%	Av inversens natur f�ljer att om \(b\) �r inversen till \(a\), d� �r �ven
%	\(a\) inversen till \(b\).
%\end{remark}

Vi ska i detta kapitel tillf�ra denna egenskap till additionen f�r de naturliga
talen.
Resultatet �r vad som kallas de hela talen, och m�ngden av de hela talen
betecknas\footnote{Anledningen till att de betecknas med \(\Z\) �r tyskans
\emph{Zahlen} som betyder tal (pluralis, j�mf�r engelskans \emph{numbers}).}
\(\Z\).


%%%%%%%%%%%%%%%%%%%%%%%%%%%%%%%%%%%
% UT�KNINGEN
%%%%%%%%%%%%%%%%%%%%%%%%%%%%%%%%%%%
\section[Ut�kningen av de naturliga talen]
{Ut�kningen av de naturliga talen\maybeskip}
\label{sec:HeltalensKonstruktion}
\noindent
\lettrine{L}{�t oss nu} skapa de hela talen.
Det enda vi har att tillg� �r de naturliga talen, dessa vet vi att de
existerar och hur de fungerar.
Vi b�rjar med att l�ta m�ngden \(M = \N\times\N =
\{(a,b) : a\text{ och }b\text{ �r naturliga tal}\}\) vara m�ngden av alla
ordnade par av naturliga tal.
\begin{example}
	Vi har exempelvis att \((1,2)\in M\) och \((2,1)\in M\) samt att
	\((1,2)\neq (2,1)\).
\end{example}

L�t oss nu inf�ra en relation \(\sim\) p� denna m�ngd.
Om \((a,b)\) och \((c,d)\) �r ordnade par av naturliga tal, det vill s�ga
element i m�ngden \(M\), d� vi vill att \((a,b)\sim (c,d)\) ska vara sant
precis n�r \(a+d=b+c\).
Denna relation \(\sim\) �r faktiskt en ekvivalensrelation.
\begin{proof}
	Vi kollar att relationen uppfyller axiomen i
	\defref{def:Ekvivalensrelation} f�r en ekvivalensrelation.
	Vi har att \((a,b)\sim (a,b)\) eftersom att \(a+b=b+a\), och allts� �r den
	reflexiv.
	Om \((a,b)\sim (c,d)\) d� �r \(a+d=b+c\) och allts� \((c,d)\sim (a,b)\)
	eftersom att \(c+b=b+c\) och \(d+a=a+d\) och allts� \(c+b=d+a\).
	S�ledes �r relationen symmetrisk.
	Om \((a,b)\sim (c,d)\) och \((c,d)\sim (e,f)\), d� har vi f�rst att
	\begin{equation}
		\label{eq:VisaHeltalensEkvivalensrelation1}
		a+d = b+c
	\end{equation}
	och sedan att
	\begin{equation}
		\label{eq:VisaHeltalensEkvivalensrelation2}
		c+f=d+e.
	\end{equation}
	Om vi drar bort \(a\) fr�n \eqref{eq:VisaHeltalensEkvivalensrelation1} f�r
	vi att \(d=b+c-a\), om vi drar bort \(b\) fr�n samma ekvation f�r vi
	\(a+d-b=c\).
	Om vi anv�nder dessa resultat i \eqref{eq:VisaHeltalensEkvivalensrelation2}
	f�r vi
	\begin{equation*}
		(a+d-b)+f = (b+c-a)+e.
	\end{equation*}
	Om vi tar bort parenteserna och flyttar om termerna f�r vi
	\begin{align*}
		a+d-b+f &= b+c-a+e, \\
		(a+b)+a+d-b+f &= (a+b)+b+c-a+e, \\
		a+a+d+f &= b+b+c+e. \\
	\end{align*}
	Men \(a+d=b+c\) enligt \eqref{eq:VisaHeltalensEkvivalensrelation1}, allts�
	f�r vi att \(a+f=b+e\) och s�ledes att \((a,b)\sim (e,f)\).
	Eftersom att relationen d� �ven �r transitiv �r relationen en
	ekvivalensrelation.
\end{proof}

Vi betecknar ekvivalensklasserna p� f�ljande s�tt:
Om \((a,b)\) �r ett element i \(M\), d� �r
\([(a,b)]=\{(x,y)\in M : (a,b)\sim (x,y)\}\).
Det vill s�ga, \([(a,b)]\) inneh�ller alla talpar i \(M\) som uppfyller
ekvivalensrelationen med talparet \((a,b)\).
\begin{example}
	Vi har att \((0,0)\sim (1,1)\) eftersom att \(0+1 = 0+1\).
	Allts� har vi \([(0,0)]=[(1,1)]\) och \((1,1)\in [(0,0)]\).
\end{example}
\begin{remark}
	Kom ih�g att det inte spelar n�gon roll om vi v�ljer \((0,0)\) eller
	\((1,1)\) som representant f�r
	ekvivalensklassen \([(0,0)]=[(1,1)]\) eftersom att det �r samma
	ekvivalensklass.
\end{remark}
\begin{example}
	Vi har d�remot \((0,0)\nsim (1,0)\) eftersom att \(0+0\neq 0+1\).
	Allts� har vi \((1,0)\notin [(0,0)]\).
\end{example}

L�t oss nu definiera en bin�r operation \(+\) som vi kallar f�r addition och en
bin�r operation \(\cdot\) som vi kallar f�r multiplikation p� m�ngden av
ekvivalensklasser hos \(M\).
Vi l�ter \([(a,b)]+[(c,d)] = [(a+c, b+d)]\) och \([(a,b)]\cdot [(c,d)] =
[(ac+bd, ad+bc)]\).
Vi m�ste dock visa att dessa �r v�ldefinierade.
\begin{proof}[V�ldefinierad addition]
	Om \((a,b)\sim (a^\prime, b^\prime)\) och \((c,d)\sim (c^\prime,
	d^\prime)\), d� vill vi visa att \((a+c, b+d)\sim (a^\prime+c^\prime,
	b^\prime+d^\prime)\).
	Vi har fr�n \((a,b)\sim (a^\prime,b^\prime)\) att \(a+b^\prime =
	b+a^\prime\) och fr�n \((c,d)\sim (c^\prime,d^\prime)\) att
	\(c+d^\prime=d+c^\prime\).
	Om vi adderar v�nsterleden i de b�da ekvationerna, d� m�ste detta vara lika
	med summan av h�gerleden.
	Allts�, \((a+b^\prime) + (c+d^\prime) = (b+a^\prime) + (d+c^\prime)\).
	Om tar bort parenteserna och byter plats p� termerna f�r vi att
	\begin{equation*}
		a+c+b^\prime+d^\prime = b+d+a^\prime+c^\prime,
	\end{equation*}
	och allts� m�ste \((a+c,b+d)\sim (a^\prime+c^\prime, b^\prime+d^\prime)\).
\end{proof}
%\begin{proof}[V�ldefinierad multiplikation]
	% TODO visa v�ldef multiplikation
%	multiplicera ekvivalensekvationerna med varandra likt de adderades i ovan.
%\end{proof}
\begin{exercise}
	Visa att multiplikationen �r v�ldefinierad.
\end{exercise}

\begin{example}
	\label{ex:HeltalAddition}
	Om vi adderar \([(1,2)]\) och \([(2,4)]\) f�r vi
	\begin{equation*}
		[(1,2)]+[(2,4)] = [(1+2,2+4)] = [(3, 6)] = [(0,3)].
	\end{equation*}
\end{example}
\begin{example}
	\label{ex:HeltalMultiplikation}
	Om vi multiplicerar \([(1,2)]\) och \([(2,4)]\) f�r vi
	\begin{equation*}
		[(1,2)]\cdot [(2,4)] = [(1\cdot 2 + 2\cdot 4, 1\cdot 4 + 2\cdot 2)] =
		[(10, 8)] = [(2,0)].
	\end{equation*}
\end{example}

Vi har nu tal med egenskaperna att de inneh�ller de naturliga talen och vi kan
addera dem f�r att f� identitetselementet.
Identitetselementet f�r addition m�ste vara ekvivalensklassen \([(0,0)]\).
\begin{proof}
	Eftersom att \([(a,b)]+[(0,0)] = [(a+0,b+0)] = [(a,b)]\) och
	\([(0,0)]+[(a,b)] = [(0+a,0+b)] = [(a,b)]\) m�ste \([(0,0)]\)
	vara identitetselementet f�r addition.
\end{proof}

Inversen till \([(a,b)]\) �r \([(b,a)]\).
\begin{proof}
	Inversen till \([(a,b)]\) �r \([(b,a)]\) eftersom att
	\([(a,b)]+[(b,a)]=[(a+b,b+a)]=[(a+b,a+b)]\).
	Eftersom att \(0+(a+b)=0+(a+b)\) har vi enligt ekvivalensrelationen att
	\((0,0)\sim (a+b,a+b)\) och allts� m�ste de tillh�ra samma ekvivalensklass.
	Detta inneb�r att \([(a,b)]\) och \([(b,a)]\) m�ste vara varandras inverser.
\end{proof}
Allts� har varje tal �tminstone en invers, vilket var vad vi �nskade.
L�t oss beteckna inversen f�r heltalet \(a\) med \(-a\).

\begin{exercise}
	Kan ett tal ha fler �n en invers?
\end{exercise}

L�t oss nu inf�ra f�ljande enklare beteckningar.
\begin{definition}
	L�t \(0\) beteckna ekvivalensklassen \([(0,0)]\).
	L�t \(1\) beteckna ekvivalensklassen \([(1,0)]\), \(2\) beteckna
	\([(2,0)]\), och \(a\) beteckna \([(a,0)]\).
\end{definition}
De nya beteckningarna och ekvivalensklasserna visas i
\figref{fig:HeltalensEkvivalensklasser}.

\begin{exercise}
	I \exref{ex:HeltalAddition}, vilka tal adderades och vilket blev resultatet?
\end{exercise}
\begin{exercise}
	I \exref{ex:HeltalMultiplikation}, vilka tal multiplicerades och vilket
	blev resultatet?
\end{exercise}

\begin{figure}[t]
	\centering
	\includegraphics[width=8cm]{Integers_Representation.png}
	\caption{Ekvivalensklasser f�r par av naturliga tal \((n_1,n_2)\) under
		relationen \(\sim\).}
	\label{fig:HeltalensEkvivalensklasser}
\end{figure}

Vi sammanfattar detta avsnitt med f�ljande definitioner.
Vi b�rjar med att definiera vad de hela talen �r.
\begin{definition}[Heltalen]
	\index{heltal}
	L�t \(M=\N\times\N\) vara m�ngden av alla naturliga talpar och \(\sim\)
	vara ekvivalensrelationen p� m�ngden \(M\) s�dan att \((a,b)\sim (c,d)\)
	�r uppfylld f�r element \((a,b)\) och \((c,d)\) i \(M\) om \(a+d=b+c\).

	L�t \(\Z = M/\!\sim = \{[(a,b)] : (a,b)\in M\}\) vara m�ngden av alla
	ekvivalensklasser f�r \(M\) under ekvivalensrelationen \(\sim\).
	Varje element \(n\) i \(\Z\) kallar vi ett \emph{heltal}.

	L�t ocks� \(0\) beteckna \([(0,0)]\in\Z\), \(1\) beteckna \([(1,0)]\in\Z\)
	och \(a\) beteckna \([(a,0)]\).
\end{definition}

N�r vi nu vet vad ett heltal �r kan det vara passande att inf�ra aritmetiska
operationer f�r dem.
Vi b�rjar med additionen, vi tar multiplikationen senare.
\begin{definition}[Addition]
	\index{addition}\index{heltal!addition}
	L�t \(x=[(a,b)]\) och \(y=[(c,d)]\) vara heltal och \(+\) en bin�r
	operation p� \(\Z\), kallad \emph{addition}, s�dan att
	\(x+y=[(a,b)]+[(c,d)]\) �r lika med \([(a+c,b+d)]\).
\end{definition}

Nu n�r vi kan addera heltal �r det l�mpligt att vi tittar p� hur vi kan j�mf�ra
dem annat �n med likhet.
Olikheter inf�rdes med hj�lp av additionen f�r de naturliga talen, det �r inte
f�rv�nande att vi g�r detsamma f�r heltalen.
\begin{definition}[Olikhet]
	\index{olikhet}
	L�t \(x=[(a,b)]\) och \(y=[(c,d)]\) vara heltal.
	Vi s�ger att \(x\) \emph{�r mindre �n} \(y\), betecknat \(x<y\), om
	\(a+d<b+c\) enlingt relationen \(<\) p� de naturliga talen.
	Vi kan ocks� s�ga att \(y\) \emph{�r st�rre �n} \(x\) och skriver d�
	\(y>x\).
	Vi s�ger att \(x\) \emph{�r mindre �n eller lika med} \(y\) om \(a+d\leq
	b+c\), detta betecknas \(x\leq y\).
	Vi kan ocks� s�ga att \(y\) \emph{�r st�rre �n eller lika med} \(x\) och
	betecknas \(y\geq x\).
\end{definition}

Nu n�r vi kan ordna de hela talen �r den naturliga efterf�ljande definitionen
denna.
\begin{definition}
	\index{negativt tal}\index{positivt tal}
	L�t oss kalla ett tal mindre �n noll f�r ett \emph{negativt} tal, och l�t
	oss kalla ett tal st�rre �n noll f�r ett \emph{positivt} tal.
\end{definition}
Vi har nu inf�rt olika namn f�r talen p� b�da sidorna om talet noll.
Nu till inverserna.
\begin{definition}
	\index{negation}\index{additiv invers}
	Om \(x=[(a,b)]\) �r ett heltal, d� �r \(-x=-[(a,b)]=[(b,a)]\) dess
	\emph{additiva invers}.
	Denna kallas �ven f�r \emph{negationen} av \(x\).
\end{definition}
\begin{remark}
	L�t \(a\) och \(b\) vara heltal och \(-b\) den additiva inversen till
	\(b\).
	Normalt skriver vi \(a+(-b)\) som \(a-b\) f�r att spara in p� n�gra tecken.
	Vi ben�mner \(a-b\) som \emph{subtraktionen}\index{subtraktion} av \(b\)
	fr�n \(a\).
	Ur en matematisk synvinkel �r subtraktion inte en egen operation utan vi
	adderar den additiva inversen f�r \(b\) till \(a\).
\end{remark}

Den andra aritmetiska operationen, multiplikation, ger vi i den sista
definitionen.
\begin{definition}[Multiplikation]
	\index{multiplikation}\index{heltal!multiplikation}
	L�t \(x=[(a,b)]\) och \(y=[(c,d)]\) vara heltal och \(\cdot\) en bin�r
	operation p� \(\Z\), kallad \emph{multiplikation}, s�dan att
	\(xy=[(a,b)]\cdot [(c,d)]\) �r lika med \([(ac+bd,ab+bc)]\).
\end{definition}

Vi ska nu visa att lemma som vi kommer att beh�va i kommande avsnitt.
Lemmat talar om att produkten av ett helt tal och noll blir noll.
\begin{lemma}
	\label{lem:IngenNolldelare}
	F�r alla heltal \(a\) g�ller att \(a\cdot 0 = 0\).
\end{lemma}
\begin{proof}
	Vi har per definition att \(a=[(a,0)]\), s� \(a\cdot 0 = [(a,0)]\cdot
	[(0,0)] = [(a\cdot 0 + 0\cdot 0, a\cdot 0 + 0\cdot 0)] = [(0,0)] = 0\).
\end{proof}

\begin{exercise}
	�r additionen kommutativ f�r de hela talen?
\end{exercise}
\begin{exercise}
	�r additionen associativ f�r de hela talen?
\end{exercise}
\begin{exercise}
	�r multiplikationen kommutativ f�r de hela talen?
\end{exercise}
\begin{exercise}
	�r multiplikationen associativ f�r de hela talen?
\end{exercise}
\begin{exercise}
	�r multiplikationen distributiv �ver additionen f�r de hela talen?
\end{exercise}
\begin{exercise}
	De naturliga talens ordning �r \(0<1<2<3<\ldots\), och s� vidare.
	Hur ordnas de hela talen?
\end{exercise}


%%%%%%%%%%%%%%%%%%%%%%%%%%%%%%%%%%%
% ALGEBRAISKA EGENSKAPER
%%%%%%%%%%%%%%%%%%%%%%%%%%%%%%%%%%%
\section{Algebraiska egenskaper f�r de hela talen}
\label{sec:HeltalensAlgebraiskaEgenskaper}
\noindent
\lettrine{D}{et �r nu dags} att sammanfatta de algebraiska egenskaperna f�r de
hela talen.
De hela talen bygger p� de naturliga talen och ska vara en ut�kning av dessa.
Vi hade f�r avsikt att de hela talen skulle ha samma algebraiska egenskaper som
vi visat att de naturliga talen har.
Vi forts�tter d�rf�r h�rn�st med en sammanfattning av de algebraiska egenskaper
som de hela talen m�ste ha.
Dessa �r desamma som f�r de naturliga talen f�rutom till�gget om att varje tal
\(a\) har en additiv invers \(-a\).

\theoremstyle{plain}
\newtheorem*{AlgebraicPropertiesIntegers}{Algebraiska egenskaper f�r de hela
	talen}
\begin{AlgebraicPropertiesIntegers}
	\nocite{Bartle2000itr,Grillet2007aa}
	\label{def:HeltalenEgenskaper}
	P� m�ngden \(\Z\) av hela tal definieras tv� bin�ra operationer,
	addition (\(+\)) och multiplikation (\(\cdot\)).
	F�r addition g�ller f�ljande:
	\begin{description}
		\item[Kommutativitet] \(a+b=b+a\) f�r alla \(a,b\in\Z\).
		\item[Associtivitet] \((a+b)+c=a+(b+c)\) f�r alla \(a,b,c\in\Z\).
		\item[Additivt identitetselement] Det finns ett element \(0\in\Z\)
			s�dant att f�r alla \(a\in\Z\) g�ller att \(0+a = a+0 = a\).
		\item[Additiv invers] F�r alla \(a\in\Z\) finns ett element \(-a\in\Z\)
			s�dant att \(a + (-a) = (-a) + a = 0\).
	\end{description}
	F�r multiplikation g�ller f�ljande:
	\begin{description}
		\item[Kommutativitet] \(a \cdot b=b \cdot a\) f�r alla \(a,b\in\Z\).
		\item[Associtivitet] \((a \cdot b) \cdot c=a \cdot (b \cdot c)\) f�r
			alla \(a,b,c\in\Z\).
		\item[Multiplikativt identitetselement] Det finns ett element
			\(1\in\Z\) s�dant att f�r alla \(a\in\Z\) g�ller att
			\(1 \cdot a = a \cdot 1 = a\).
	\end{description}
	Ut�ver detta g�ller �ven
	\begin{description}
		\item[Multiplikativ distributivitet �ver addition]
			\(a \cdot (b+c) = (a \cdot b) + (a \cdot c)\) och
			\((b+c) \cdot a = (b \cdot a) + (c \cdot a)\) f�r alla hela tal
			\(a,b,c\in\Z\).
	\end{description}
\end{AlgebraicPropertiesIntegers}



%%%%%%%%%%%%%%%%%%%%%%%%%%%%%%%%%%%%%%%%%%%%
% ALGEBRAISKA EGENSKAPER F�R DE NEGATIVA TALEN
%%%%%%%%%%%%%%%%%%%%%%%%%%%%%%%%%%%%%%%%%%%%
\section{Algebraiska egenskaper f�r de negativa talen}
\noindent
\lettrine{V}{i �r redan} bekanta med de naturliga talen, och allts� med den
delen av de hela talen som motsvarar de naturliga talen.
D�rf�r ska vi i detta avsnitt fokusera p� de nya talen, de heltal \(a<0\) som
�r mindre �n noll -- eller de negativa talen.

L�t oss b�rja med ett av de fundamentala resultaten,
% (-1)a = -a
n�mligen att ett heltal \(a\) multiplicerat med \(-1\) ger dess invers \(-a\).
Vi formulerar detta som en sats.
\begin{theorem}
	\label{thm:FaktoriseraNegativaTal}
	L�t \(a\) vara ett heltal.
	D� �r \(a\) multiplicerat med \(-1\) inversen \(-a\) till \(a\).
	Allts�, \((-1)\cdot a = -a\).
\end{theorem}
\begin{proof}
	Vi har att \(1+(-1) = 0\).
	Om vi multiplicerar noll med ett tal f�r vi enlingt
	\lemref{lem:IngenNolldelare} fortfarande noll.
	Allts� f�r vi att \(a(1+(-1))=0\) eftersom att vi d� multiplicerar \(a\)
	med noll.
	Eftersom att multiplikation �r distributiv �ver addition f�r vi d� att
	\begin{equation*}
		a(1+(-1)) = 1\cdot a + (-1)a = a + (-1)a = 0.
	\end{equation*}
	D� m�ste \((-1)a\) vara inversen \(-a\) till \(a\).
\end{proof}
\begin{remark}
	Notera att detta och f�ljande resultat �ven kan h�rledas direkt fr�n
	talparskonstruktionen i \secref{sec:HeltalensKonstruktion} trots att vi
	i detta bevis v�ljer att anv�nda oss av de algebraiska egenskaperna givna i
	\secref{sec:HeltalensAlgebraiskaEgenskaper}.
\end{remark}

% (-1)(-1) = (-1)^2 = 1
Vi forts�tter med kanske det mest h�pnanv�ckande resultatet, som �r en
f�ljdsats till f�reg�ende.
\begin{corollary}
	Den additiva inversen \(-1\) till heltalet ett multiplicerad med sig
	sj�lv �r ett.
	Det vill s�ga \((-1)(-1) = 1\).
\end{corollary}
\begin{proof}
	Eftersom att \(-1\) �r ett heltal f�ljer det fr�n
	\thmref{thm:FaktoriseraNegativaTal} att om vi multiplicerar det med \(-1\)
	f�r vi dess invers.
	Inversen f�r \(-1\) �r \(1\) och allts� har vi att \((-1)(-1)=1\).
\end{proof}

L�t oss avsluta med ett exempel om addition av tv� negativa tal.
\begin{example}
	% (-1)+(-1) = -2
	Om vi adderar \(-1\) och \(-1\), vad f�r vi d�?
	Eftersom att vi adderar tv� lika termer kan vi skriva additionen som
	multiplikationen \(2\cdot (-1) = (-1)+(-1)\).
	Vi vet fr�n \thmref{thm:FaktoriseraNegativaTal} att ett heltal
	multiplicerat med \(-1\) �r dess invers, allts� f�r vi \((-1)+(-1)=2\cdot
	(-1) = -2\).
\end{example}
\begin{example}
	Vi kan ocks� ber�kna summan \((-1)+(-1)\) genom att anv�nda heltalens
	distributiva egenskap.
	Eftersom att \((-1)=(-1)\cdot 1\) och att multiplikation �r distributiv
	�ver addition f�r vi att \((-1)+(-1) = (-1)(1+1) = (-1)(2) = -2\).
\end{example}
\begin{exercise}
	% (-1)(a+b) = (-a)+(-b)
	Om \(a\) och \(b\) �r heltal,
	g�ller generellt att \((-a)+(-b)=-(a+b)\)?
\end{exercise}

\begin{exercise}
	% (-1)(a-b) = (-1)(a+(-b)) = (-a)+b
	Utred vad \(-(a-b)\) egentligen betyder.
\end{exercise}


%%%%%%%%%%%%%%%%%%%%%%%%%%%%%%%%%%%%%%%%%%%%
% POTENSER
%%%%%%%%%%%%%%%%%%%%%%%%%%%%%%%%%%%%%%%%%%%%
\subsection{Potenser f�r heltalen}
\noindent
Vi ska nu inf�ra potenser f�r de hela talen, likt de vi inf�rde f�r de
naturliga talen.
Vi definierar potenser enligt f�ljande definition.
\begin{definition}
	L�t \(a\) vara ett heltal och l�t \(n\) vara ett naturligt tal.
	Vi skriver d�
	\begin{equation*}
		\underbrace{a\cdot a\cdots a}_n
	\end{equation*}
	som \(a^n\) och utl�ser det som en \(a\)-\emph{potens} med
	\emph{exponenten} \(n\).
\end{definition}

\begin{exercise}
	Utforska potenser f�r de hela talen.
	Vad finns det f�r intressanta resultat?
\end{exercise}

%%%%%%%%%%%%%%%%%%%%%%%%%%%%%%%%%%%%%%%%%%%%
% PRIMTAL OCH DELBARTHET
%%%%%%%%%%%%%%%%%%%%%%%%%%%%%%%%%%%%%%%%%%%%
\chapter{Talteori}
\label{ch:Talteori}
\nocite{Laksov2005kou,Bartle2000itr}
\noindent
...

%%%%%%%%%%%%%%%%%%%%%%%%%%%%%%%%%%%%%%%%%%%%
% DELBARTHET
%%%%%%%%%%%%%%%%%%%%%%%%%%%%%%%%%%%%%%%%%%%%
\subsection{Delbarhet}
\noindent
...

\begin{definition}[Delbarhet]
	...
\end{definition}

\begin{theorem}[Divisionsalgoritmen]
	...
\end{theorem}

\begin{exercise}
	\label{xrc:KvotenMindre}
	Bevisa f�ljande.
	L�t \(a\) och \(b\) vara tv� heltal.
	Kvoten \(q=a/b\) �r strikt mindre �n \(a\) om \(b>1\) �r strikt st�rre �n
	ett.
\end{exercise}


%%%%%%%%%%%%%%%%%%%%%%%%%%%%%%%%%%%%%%%%%%%%
% PRIMTAL OCH SAMMANSATTA TAL
%%%%%%%%%%%%%%%%%%%%%%%%%%%%%%%%%%%%%%%%%%%%
\subsection{Primtal och sammansatta tal}
\noindent
...

\begin{definition}[Primtal och sammansatta tal]
	...
\end{definition}

F�ljande sats, aritmetikens fundamentalsats, visades i princip av Euklides av
Alexandria (levde omkring 300 f.Kr.).
I den sjunde boken av hans \emph{Elementa} finns tv� satser som tillsammans
kan visa aritmetikens fundamentalsats.
Den visades dock f�rst i sin helhet av Carl Friedrich Gauss (1777-1855) i sin
bok \emph{Disquisitiones Arithmeticae}, som �r latin f�r \emph{utforskning av
tal}.
Gauss skrev boken 1798 n�r han var 20 �r gammal och den publicerades 1801.
Den handlar om talteori och sammanfattar tidigare resultat men introducerar
�ven nya, aritmetikens fundamentalsats var ett av dem.
\begin{theorem}[Aritmetikens fundamentalsats]
	\label{thm:AritmetikensFundamentalsats}
	Ett heltal \(n\in\Z\) kan skrivas som en unik produkt av primtal och \(1\)
	eller \(-1\).
\end{theorem}
\begin{proof}
	...
\end{proof}

F�ljande sats har varit k�nd under v�ldigt l�ng tid, den �r idag k�nd som
Euklides sats.
Eventuellt var satsen k�nd �ven tidigare, men den skrevs ned av Euklides i
den nionde boken av \emph{Elementa}.
Euklides \emph{Elementa} bestod av totalt 13 b�cker och anv�ndes som l�robok
i matematik �nda fram till 1900-talet.
\begin{theorem}[Euklides sats]
	Det finns o�ndligt m�nga primtal.
\end{theorem}
\begin{proof}
	Om vi antar att det finns �ndligt m�nga primtal, d� kan vi anta att
	\(S=\{p_1,p_2,\ldots,p_n\}\) �r m�ngden av alla primtal.
	Vi tittar p� ett heltal \(m\).
	Enligt aritmetikens fundamentalsats,
	\thmref{thm:AritmetikensFundamentalsats}, kan vi v�lja detta \(m\) s�dant
	att \(m=p_1\cdots p_n\) �r produkten av alla primtal.
	Vi l�ter dessutom \(q=m+1\).
	Eftersom att \(q>p_i\) f�r alla \(i\) kan inte \(q\) vara ett element i
	\(S\), och d�rf�r �r \(q\) inte ett primtal.
	D� finns det ett primtal \(p\) som delar \(q\).
	Eftersom att \(p\) �r ett primtal m�ste \(p=p_j\) f�r n�got \(j\), och
	allts� m�ste \(p\) dela \(m\).
	Men om \(p\) delar b�de \(m\) och \(q=m+1\), d� m�ste \(p\) �ven dela
	\(q-m=1\).
	D� detta �r om�jligt f�r vi en mots�gelse och det m�ste allts� finnas
	o�ndligt m�nga primtal.
\end{proof}



%\chapter{De reella talen\maybeskip[Kapitlet �r med enbart f�r att visa
dispositionen.]}
\label{ch:Reella}
\nocite{Kline1990mtf3}
\nocite{KTHCirkel2005rt}
\noindent
...

%%%%%%%%%%%%%%%%%%%%%%%%%%%%%%%%%%%%%
% DEDEKINDS SNITT
%%%%%%%%%%%%%%%%%%%%%%%%%%%%%%%%%%%%%
\section{Dedekinds snitt}
\noindent
...

\begin{definition}[Snitt]
	...
\end{definition}


%%%%%%%%%%%%%%%%%%%%%%%%%%%%%%%%%%%%%
% ARITMETIK
%%%%%%%%%%%%%%%%%%%%%%%%%%%%%%%%%%%%%
\section{Aritmetik}
\noindent
...

\begin{definition}[Algebraiska egenskaper hos de reella talen]
	P� m�ngden \(\R\) definieras tv� bin�ra operatorer, addition (\(+\)) och
	multiplikation (\(\cdot\)).
	F�r addition g�ller f�ljande:
	\begin{description}
		\item[Kommutativitet] \(a+b=b+a\) f�r alla \(a,b\in\R\).
		\item[Associtivitet] \((a+b)+c=a+(b+c)\) f�r alla \(a,b,c\in\R\).
		\item[Additiv enhet] Det finns ett element \(0\in\R\) s�dant att
			f�r alla \(a\in\R\) g�ller att \(0+a = a+0 = a\).
		\item[Additiv invers] F�r alla \(a\in\R\) finns ett element \(-a\in\R\)
			s�dant att \(a + (-a) = (-a) + a = 0\).
	\end{description}
	F�r multiplikation g�ller f�ljande:
	\begin{description}
		\item[Kommutativitet] \(a \cdot b=b \cdot a\) f�r alla \(a,b\in\R\).
		\item[Associtivitet] \((a \cdot b) \cdot c=a \cdot (b \cdot c)\) f�r
			alla \(a,b,c\in\R\).
		\item[Multiplikativ enhet] Det finns ett element \(1\in\R\) s�dant att
			f�r alla \(a\in\R\) g�ller att \(1 \cdot a = a \cdot 1 = a\).
		\item[Multiplikativ invers] F�r alla \(0 \neq a\in\R\) finns ett
			element \(1/a\in\R\) s�dant att
			\(a \cdot (1/a) = (1/a) \cdot a = 1\).
	\end{description}
	Ut�ver detta g�ller �ven
	\begin{description}
		\item[Multiplikativ distribuitet �ver addition]
			\(a \cdot (b+c) = (a \cdot b) + (a \cdot c)\) och
			\((b+c) \cdot a = (b \cdot a) + (c \cdot a)\) f�r alla reella tal
			\(a,b,c\in\R\).
	\end{description}
\end{definition}

\begin{exercise}
	Utforska vad de algebraiska egenskaperna hos de reella talen till�ter.
\end{exercise}
\begin{exercise}
	Om \(x\in\R\) och \(a\in\R\) b�da �r reella tal och \(x + a = a\),
	visa att \(x=0\).
\end{exercise}
\begin{exercise}
	Om \(x\in\R\) och \(a\in\R\) b�da �r reella tal och \(a \cdot x = a\),
	visa att \(x=1\).
\end{exercise}
\begin{exercise}
	Om \(a\in\R\) �r ett reellt tal, visa att \(a \cdot 0 = 0\).
\end{exercise}
\begin{exercise}
	Om \(0 \neq a \in \R\) och \(b\in\R\) �r reella tal och \(a \cdot b = 1\),
	visa att \(b = 1/a\).
\end{exercise}
\begin{exercise}
	Om \(a\in\R\) och \(b\in\R\) �r reella tal och \(a \cdot b = 0\),
	visa att antingen \(a=0\) eller \(b=0\), eller b�da.
\end{exercise}


%%%%%%%%%%%%%%%%%%%%%%%%%%%%%%%%%%%%%%%%%%
% POTENSER
%%%%%%%%%%%%%%%%%%%%%%%%%%%%%%%%%%%%%%%%%%
\subsection{Potenser}
\noindent
...

%{F�rkunskaper: m�ngder, heltalsdivision, potenser}
\chapter{Talsystem}\index{talsystem}\index{talbeteckningssystem|see{talsystem}}
\label{ch:Talsystem}
\noindent
\lettrine{V}{i vet sedan} tidigare avsnitt att tal existerar och att det finns
olika typer av tal; naturliga tal, hela tal, rationella tal och irrationella
tal.
Vi vet dessutom att det finns o�ndligt\footnote{Med detta �r det inte sagt att
de olika m�ngderna har samma kardinalitet.} m�nga tal av varje typ samt att de
g�r att r�kna med p� olika s�tt.
Men vad �r egentligen ett tal?

Ett tal �r inom matematiken ett \emph{abstrakt objekt} som f�ljer givna regler,
exempelvis de f�r heltalen givna reglerna som vi behandlade i
\chref{ch:Heltalen}.
�r d� \(123\) ett tal?
Nej, \(123\) �r bara en \emph{representation} av talet vi \emph{ben�mner}
etthundratjugotre.
Det b�r p�pekas att ordet etthundratjugotre ocks� �r en representation av
talet, och allts� inte talet sj�lvt.
Ett talsystem\index{talsystem}, eller
talbeteckningssystem\index{talbeteckningssystem}, tillhandah�ller ett entydigt
s�tt att representera dessa abstrakta tal p�.
Genom historien har det anv�nts m�nga olika talsystem, varav n�gra finns kvar
�n idag.
I \tblref{tbl:OlikaEtthundratjugotre} ges n�gra representationer av
etthundratjugotre i olika talsystem.

\begin{table}[h]
	\caption{Olika representationer av etthundratjugotre i olika talsystem.}
	\begin{tabular}{ll}
		\hline\hline
		Bin�ra talsystemet & \(1111011\) \\
		Decimala talsystemet & \(123\) \\
		Hexadecimala talsystemet & \(7B\) \\
		Romerska talsystemet & LXXIII \\
		\hline\hline
	\end{tabular}
	\label{tbl:OlikaEtthundratjugotre}
\end{table}

\begin{exercise}
	\label{xrc:HittaTalsystem}
	Fundera p� alla s�tt du k�nner till att representera tal p�.
\end{exercise}
\begin{exercise}
	\label{xrc:SkapaTalsystem}
	F�rs�k att hitta p� ett eget s�tt att representera tal p�.
\end{exercise}

Det talsystem som �r vanligast idag �r det \emph{decimala talsystemet}, d�r
tecknen som anv�nds �r siffrorna \(1, 2, 3, 4, 5, 6, 7, 8\) och \(9\).
\begin{remark}
	M�rk v�l skillnaden mellan ett decimalt tal, som �r ett tal skrivet med
	det decimala talsystemet, och ett decimaltal, som �r ett
	tal med komma och decimaler.
\end{remark}
Andra vanliga talsystem �r det bin�ra, med siffrorna \(0\) och \(1\), och det
hexadecimala, med 16 olika siffror.
Dessa tv� anv�nds flitigt inom datateknik.
Det decimala, det bin�ra och det hexadecimala talsystemen �r av en speciell
typ av talsystem som kallas \emph{positionssystem}.
Anledningen till namnet �r att en siffras position har betydelse f�r dess
v�rde.
\begin{example}\label{ex:PositionensBetydelse}
I \(111\) betyder den f�rsta ettan \(100\) medan den andra ettan
betyder \(10\) och den sista betyder \(1\).
Det vill s�ga, samma siffra har olika betydelse beroende p� var i
representationen som den befinner sig.
\end{example}
Positionssystemen behandlas i detalj i kommande avsnitt.

Det romerska talsystemet �r d�remot inte ett positionssystem, utan �r en
modifikation av typen \emph{teckenv�rdessystem}.
Det romerska systemet behandlas i n�sta avsnitt.


%%%%%%%%%%%%%%%%%%%%%%%%%%%%%%%%%%%%%%%%%
% DET ROMERSKA TALSYSTEMET
%%%%%%%%%%%%%%%%%%%%%%%%%%%%%%%%%%%%%%%%%
\section{Det romerska talsystemet}
\index{romerska talsystemet}\index{talsystem!romerskt}
\noindent
\lettrine{D}{et romerska talsystemet} �r baserat p� en modifikation av ett
\emph{teckenv�rdessystem}.
I ett teckenv�rdessystem har varje tecken ett speciellt v�rde.
Detta till skillnad fr�n positionssystemet d�r positionen �r avg�rande f�r
tecknets v�rde.
I ett mycket enkelt teckenv�rdessystem, som anv�nds idag, representerar ett
streck talet ett, tv� streck representerar talet tv�, och s� vidare till och
med talet fyra.
Allts� har samma tecken samma v�rde och upprepningar adderas tillsammans f�r
att f� talet de representerar.
Talet fem representeras d�remot med fyra streck, som ovan, och ett femte streck
snett �ver de fyra andra strecken.
Detta bildar ett nytt tecken �ven om det �r logiskt uppbyggt fr�n tidigare
tecken.
Systemet best�r allts� av tv� tecken, ett tecken som har v�rdet ett och ett
tecken som har v�rdet fem.
Se \figref{fig:Strecktal}.

\begin{figure}[h]
	\caption{Tecknen i ett enkelt teckenv�rdessystem.}
	% TODO fixa figur f�r streckteckensystemet
	%\includegraphics{}
	Ingen figur �nnu ...
	\label{fig:Strecktal}
\end{figure}

Det romerska systemet �r en modifiering av detta system.
Tecknen och deras betydelse i det romerska talsystemet ges i
\tblref{tbl:RomerskaSiffror}.

\begin{table}[h]
	\caption{De romerska siffrorna.}
	\begin{tabular}{ccccccc}
		\hline\hline
		I & V & X & L & C & D & M \\
		1 & 5 & 10 & 50 & 100 & 500 & 1000 \\
%		I & 1 \\
%		V & 5 \\
%		X & 10 \\
%		L & 50 \\
%		C & 100 \\
%		D & 500 \\
%		M & 1000 \\
		\hline\hline
	\end{tabular}
	\label{tbl:RomerskaSiffror}
\end{table}

Det romerska talsystemt fungerar n�stan p� samma s�tt som ett
tecken\-v�rdes\-system.
Den v�sentliga skillnaden �r att i teckenv�rdessystemet summeras alla tecknens
v�rden medan det i det romerska systemet �ven finns subtraktion.
I det romerska systemet subtraheras ett tecken av l�gre v�rde som st�r f�re ett
tecken med h�gre v�rde.
Exempelvis, IV ger \(4\) eftersom att en etta st�r f�re en femma.
VI ger d�remot \(6\) eftersom att tecknens v�rde summeras.
Talen 1-10 ges i \tblref{tbl:RomerskaTal} och n�gra andra st�rre tal finns i
\tblref{tbl:RomerskaTal2}.

\begin{table}[h]
	\caption{Talen 1-10 i det decimala och det romerska talsystemen.}
	\begin{tabular}{l|llllllllll}
		\hline\hline
		Decimala talsystemet & 1 & 2 & 3 & 4 & 5 & 6 & 7 & 8 & 9 & 10 \\
		Romerska talsystemt & I & II & III & IV & V &
			VI & VII & VIII & IX & X \\
		\hline\hline
	\end{tabular}
	\label{tbl:RomerskaTal}
\end{table}

\begin{table}[h]
	\caption{N�gra tal skrivna med det romerska talsystemet.}
	\begin{tabular}{ll}
		\hline\hline
		\(2011\) & MMXI \\
		\(1999\) & MCMXCIX \\
		\(1998\) & MCMXCVIII \\
		\(587\) & DLXXXVII \\
		\(487\) & CDLXXXVII \\
		\hline\hline
	\end{tabular}
	\label{tbl:RomerskaTal2}
\end{table}

Som kan ses i \tblref{tbl:RomerskaTal2} �r det endast tecknet med direkt l�gre
v�rde som anv�nds vid subtraktion.
Exempelvis kan t�nkas att \(487\) �r \(500\) minus \(13\) och d�rf�r skulle
kunna skrivas som XIIID, men s�dant �r allts� inte fallet.
Det �r ist�llet C, eller \(100\), som anv�nds f�r subtraktion f�r D, eller
\(500\), och sedan l�ggs tecknen f�r \(87\) till som vanligt.

\begin{exercise}\label{xrc:RaknaMedRomerskaTal}
	Unders�k och diskutera hur enkelt det �r att r�kna med och
	representera tal med det romerska talsystemet.
\end{exercise}
\begin{exercise}
	Med utg�ngspunkt i f�reg�ende �vning, hur tror du att romarna bidragit till
	matematikhistorien?
\end{exercise}
\begin{exercise}
	Hur �r det med talet noll och negativa tal i det romerska talsystemet?
\end{exercise}



%%%%%%%%%%%%%%%%%%%%%%%%%%%%%%%%%%%%%%%%%
% POSITIONSSYSTEM
%%%%%%%%%%%%%%%%%%%%%%%%%%%%%%%%%%%%%%%%%
\section{Positionssystem}
\index{positionssystem}\index{talsystem!positionssystem}
\label{sec:Positionssystem}
\noindent
\lettrine{S}{om n�mnts tidigare} inneb�r ett positionssystem att samma tecken
har olika betydelse eller v�rde beroende p� vilken position tecknet innehar.
Systemet har en \emph{talbas}.
Det decimala talsystemet har basen \(10\) och positionen ger exponenten i en
\(10\)-potens.
Siffran p� denna position ger multipeln f�r denna potens.

\begin{example}\label{ex:DecimaltPosistionssystem}
	Det decimala talsystemet har basen \(10\).
	Talet etthundratjugotre representeras i detta system som \(123\).
	Vi har allts�
	\[1\cdot10^2 + 2\cdot10^1 + 3\cdot10^0 = 100 + 20 + 3 = 123.\]
\end{example}
\begin{example}\label{ex:BinartPositionssystem}
	Det bin�ra talsystemet har basen \(2\).
	S�ledes representeras talet etthundratjugotre som \(1111011\).
	Vi har allts�
	\[1\cdot2^6 + 1\cdot2^5 + 1\cdot 2^4 + 1\cdot2^3 + 0\cdot2^2 +
	1\cdot2^1 + 1\cdot2^0 = 123.\]
\end{example}
\begin{example}
	Om ett positionssystem har basen \(b\) anv�nds d� \(b\) antal siffror,
	dessa har v�rdena \(0, 1, 2, \ldots, b-1\).
	F�r att representera ett tal anv�nds summor av \(b\)-potenser d�r siffrans
	position avg�r exponenten.
	Siffrans v�rde avg�r multipeln f�r, det vill s�ga hur m�nga av, den
	specifika \(b\)-potensen som ska adderas.
	Det vill s�ga talet vars v�rde ber�knas som
	\[
		d_1b^{n-1} + d_2b^{n-1} + \cdots + d_{n-1}b^1 + d_nb^0
	\]
	representeras som \(d_1d_2\cdots d_n\) i basen \(b\).
\end{example}

Det finns �ven andra talsystem, exempelvis babyloniernas
sexagesimala positionssystem\index{talsystem!babylonskt}
\index{talsystem!sexagesimalt} som hade basen \(60\).
Sp�r av detta kan vi idag se i hur vi r�knar tid, att en minut har 60
sekunder och en timme har 60 minuter.
Vi anv�nder dock inte 60 olika tecken f�r att representera v�ra sekunder och
minuter utan tar ist�llet hj�lp av det decimala talsystemet.
Det babylonska sexagesimalsystemet �r det �ldsta k�nda anv�ndandet av ett
positionssystem och h�rstammar fr�n Babylonien\footnote{Babylonien l�g i
s�dra delen av Mesopotamien, ungef�r i dagens Irak.} omkring 3100 f.Kr.

Kan d� alla naturliga tal verkligen representeras med en godtycklig bas p�
detta s�tt?
F�ljande sats visar att s�dant �r fallet.
\begin{theorem}\label{thm:PositionssystemAllaUnika}
	Alla naturliga tal \(x \in \N\) kan skrivas som en summa av
	\(b\)-potenser, d�r \(b\in\N\setminus\{0,1\}\) och varje potens f�r
	f�rekomma maximalt \(b-1\) g�nger i summan.
	Denna summa �r dessutom unik upp till ordningen av termerna.
\end{theorem}
\begin{proof}
	% TODO bevisa att alla naturliga tal kan representeras med en godtycklig
	% TODO bas och att denna representation �r unik.
	Inget bevis �nnu ...
\end{proof}

\begin{exercise}
	Diskutera inneb�rden av denna sats och dess bevis.
\end{exercise}

Eftersom att vi enligt \thmref{thm:PositionssystemAllaUnika} kan representera
alla naturliga tal i en godtycklig bas och att denna representation �r unik
kan vi med s�kerhet definiera ett positionssystem enligt f�ljande.
\begin{definition}[Positionssystem\footnote{\emph{Eng.} positional system,
	place-value system}]
	\index{talsystem!positionsv�rdesystem}\index{positionsv�rdesystem}
	\index{talbas}\index{positionsv�rdesystem!talbas}
	\index{positionssystem|see{positionsv�rdesystem}}
	\label{def:Positionssystem}
	Ett \emph{positionssystem}, eller positionsv�rdesystem, har en
	\emph{talbas} \(b \in \N \setminus \{0,1\}\),
	siffrorna \(S=\{s\in\N : s<b\}\) och representerar ett tal \(x\in\N\) som
	\(d_1d_2\cdots d_n\), d�r \(d_i \in S\) �r siffran p� position \(i\), och
	\begin{equation}
		\nonumber
		x = d_1b^{n-1} + d_2b^{n-2} + \ldots + d_{n-1}b^1 + d_nb^0.
	\end{equation}
\end{definition}

\begin{example}\label{ex:DecimalaTalsystemet}
	Det decimala talsystemet har basen \(b=10\) och anv�nder siffrorna
	\(S=\{0,1,2,3,4,5,6,7,8,9\}\).
\end{example}
\begin{example}\label{ex:BinaraTalsystemet}
	Det bin�ra talsystemet har basen \(b=2\) och anv�nder siffrorna
	\(S=\{0,1\}\).
\end{example}
\begin{example}\label{ex:HexadecimalaTalsystemet}
	Det hexadecimala talsystemet har basen \(b=16\).
	N�r ett talsystem har basen \(b>10\) anv�nds vanligtvis bokst�ver fr�n
	alfabetet som siffror f�r v�rdena \(10\), \(11\) och s� vidare.
	Det hexadecimala talsystemet anv�nder allts� vanligtvis
	siffrorna \[S=\{0,1,2,3,4,5,6,7,8,9,A,B,C,D,E,F\},\] d�r \(A=10\),
	\(B=11\) och s� vidare.
\end{example}

F�r att kunna urskilja vilken talbas ett tal representeras med brukar basen
anges som ett subskript.
Exempelvis talet \(123\) skrivet med det decimala systemet anges som
\(123_{10}\).
Det blir d� l�ttare att f�rst� \(123_{10} = 1111011_2\) som betyder att \(123\)
i bas 10 skrivs som \(1111011\) i bas 2.
Vanligtvis, n�r basen �r sj�lvklar, brukar den utel�mnas.
I de f�rsta 9 �ren i grundskolan �r det uteslutande det decimala talsystemet
som anv�nds och allts� har det aldrig varit n�dv�ndigt att d�r specificera att
basen varit 10.

\begin{exercise}\label{xrc:GiltigaTalbaser}
	Enligt \defref{def:Positionssystem} anv�nds inte talen \(0\) och \(1\)
	som baser, f�rs�k att f�rklara varf�r.
\end{exercise}
\begin{exercise}\label{xrc:RationelltPositionssystem}
	Vidareutveckla \defref{def:Positionssystem} till att �ven
	omfatta rationella tal.
\end{exercise}
\begin{exercise}\label{xrc:KomplettRationelltPositionssystem}
	Visa att alla rationella tal kan representeras med en godtycklig bas
	\(1<b\in\N\) enligt den nya definitionen fr�n
	\xrcref{xrc:RationelltPositionssystem}.
\end{exercise}
\begin{exercise}\label{xrc:Decimalfoljder}
	Det rationella talet \(\frac{1}{3} = 0.333\ldots\) skrivs som en o�ndlig
	decimalutveckling i bas \(10\).
	�r det s� i alla baser \(1<b\in\N\)?
\end{exercise}
\begin{exercise}\label{xrc:Basbyte}
	F�rs�k att formulera en metod f�r att byta talbas f�r ett tal.
\end{exercise}
%\begin{exercise}\label{xrc:NumSysConvert}
%	Unders�k vad som h�nder vid heltalsdivision mellan ett tal \(x\), skrivet i
%	basen \(b\), och dess bas \(b\). Vad h�nder om talet \(x\) heltalsdivideras
%	med ett tal \(b^\prime \neq b\).
%\end{exercise}


%%%%%%%%%%%%%%%%%%%%%%%%%%%%%%%%%%%%%%%%%%%%%%%%%%%%%%%%%%%%%%
% BYTE AV TALBAS
%%%%%%%%%%%%%%%%%%%%%%%%%%%%%%%%%%%%%%%%%%%%%%%%%%%%%%%%%%%%%%
\section{Byte av talbas}
\noindent
\lettrine{E}{ftersom att} \thmref{thm:PositionssystemAllaUnika} s�ger att alla
naturliga tal kan representeras i alla baser inneb�r detta att samma tal har
en unik representation i varje bas.
Det kan allts� vara intressant att se ett tals olika representationer i olika
baser.
Hur detta basbyte g�r till och att det fungerar framg�r av beviset f�r satsen.
Metoden illustreras h�r med nedan givna exempel.
\begin{example}
	Talet \(123_{10} = 1111011_2\), detta finner vi genom f�ljande:
	\begin{align*}
		123 &= 61\cdot 2 + 1 \\
		61 &= 30\cdot 2 + 1 \\
		30 &= 15\cdot 2 + 0 \\
		15 &= 7\cdot 2 + 1 \\
		7 &= 3\cdot 2 + 1 \\
		3 &= 1\cdot 2 + 1 \\
		1 &= 0\cdot 2 + 1
	\end{align*}
	Allts� f�r vi
	\begin{multline}
		(((((((0+1)\cdot2 + 1)\cdot2 + 1)\cdot2 + 1)\cdot2 + 1) + 0)\cdot2 + 1)
			\cdot2 + 1 \\
		= 1\cdot 2^6 + 1\cdot 2^5 + 1\cdot 2^4 + 1\cdot^3 + 0\cdot 2^2 +
			1\cdot 2^1 + 1\cdot 2^0 \\
		= 1111011_2 = 123_{10}.
	\end{multline}
\end{example}
\begin{example}
	Talet \(123_{10} = 7B_{16}\), detta finner vi genom f�ljande:
	\begin{align*}
		123 &= 7\cdot16 + 11 \\
		7 &= 0\cdot16 + 7
	\end{align*}
	Allts� f�r vi siffrorna \(7\) och \(B\) samt att
	\begin{equation*}
		(0+7)\cdot16 + 11 = 7\cdot16^1 + 11\cdot16^0 = 7B_{16} = 123_{10}.
	\end{equation*}
\end{example}

\begin{exercise}\label{xrc:TalbasDatavetenskap}
	Inom datateknik �r baserna \(2=2^1\), \(8=2^3\) och \(16=2^4\) v�ldigt
	popul�ra, men bas \(10=2\cdot5\) �r d�remot inte lika popul�r.
	Datorns interna representation av tal sker i form av \emph{bitar} som kan
	anta v�rdena \emph{p�} och \emph{av}, eller \(1\) och \(0\).
	Detta motsvarar precis det bin�ra talsystemet, det vill s�ga basen
	\(2=2^1\), detta f�rklarar basens popularitet.
	F�rs�k att f�rklara varf�r baserna \(8=2^3\) och \(16=2^4\) �r popul�ra,
	medan bas \(10=2\cdot5\) inte �r det.
\end{exercise}



%%%%%%%%%%%%%%%%%%%%%%%%%%%%%%%%%%%%%%%%%%%%%%%%%%%%%%%%%%%%%%
% EN ADDITIONSALGORITM
%%%%%%%%%%%%%%%%%%%%%%%%%%%%%%%%%%%%%%%%%%%%%%%%%%%%%%%%%%%%%%
\section{En additionsalgoritm}
\noindent
\lettrine{P}{ositionens betydelse} f�r siffrornas v�rde i positionssystemet g�r
att tal representerade i detta talsystem blir v�ldigt enkla att r�kna med.
Vi ska i detta avsnitt unders�ka varf�r.

Om vi vill addera de tv� talen \(x\) och \(y\) representerade med samma talbas
\(b\) kan vi g�ra enligt f�ljande.
Vi adderar v�rdena f�r siffror p� samma position i de b�da representationerna.
T�nk p� att om talen representeras med olika antal siffror kan det kortare
talet kompletteras med inledande nollor.
Dessa summor \(s_i\) kan ...

\begin{exercise}\label{xrc:MultiplikationsAlgoritm}
	Visa att den v�lk�nda multiplikationsalgoritmen, som elever l�r sig i den
	svenska grundskolan, �ven den fungerar f�r alla baser \(1<b\in\N\).
\end{exercise}


\backmatter
\ifdraft{
	\singlespacing
}{}
\bibliographystyle{kthnat}                                                      
\bibliography{../thesis/thesis}

\listoffigures
\listoftables
\clearpage
\printindex
\clearpage


% appendices
\appendix
\ifdraft{
	\doublespacing
}{}
\chapter{Kursplanering}
\noindent
\lettrine{K}{ursen Matematik 1c}
% Kursen Matematik 1c
�r enligt �mnesplanen 100 po�ng, d.v.s. den ska ges 100 lektionstimmar.
Dessa lektionstimmar ska f�rdelas p� de fem huvudomr�den som ges under rubriken
\emph{Centralt inneh�ll} i �mnesplanen, omr�dena som ges �r
\begin{itemize}
	\item Taluppfattning, aritmetik och algebra,
	\item Geometri,
	\item Samband och f�r�ndring,
	\item Sannolikhet och statistik, samt
	\item Probleml�sning.
\end{itemize}

I planeringen given nedan avser tids�tg�ngen undervisningstid, d.v.s. tid
n�r l�raren exempelvis g�r igenom inneh�ll vid tavlan.
Denna tid innefattar inte elevers redovisningar av uppgifter,
klassrumsdiskussioner etc.

Avsnittet \emph{Probleml�sning}, som inneh�ller punkterna
\begin{itemize}
	\item Strategier f�r matematisk probleml�sning inklusive anv�ndning av
		digitala medier och verktyg,
	\item Matematiska problem av betydelse f�r privatekonomi, samh�llsliv och
		till�mpningar i andra �mnen, och
	\item Matematiska problem med anknytning till matematikens kulturhistoria,
\end{itemize}
kan med f�rdel inkluderas i undervisningen som tar upp �vriga omr�den och
beh�ver s�ledes inte behandlas specifikt.

Inledande i kursen b�r vara grundl�ggande logik\footnote{Fr�n omr�det
\emph{Geometri} har vi punkten ''Matematisk argumentation ned hj�lp av
grundl�ggande logik inklusive implikation och ekvivalens samt j�mf�relser med
hur man argumenterar i vardagliga sammanhang och inom naturvetenskapliga
�mnen''.}, begreppen definition och axiom samt sats och bevis\footnote{Fr�n
omr�det \emph{Geometri} har vi punkten ''Illustration av begreppen definition,
sats och bevis, till exempel med Pythagoras sats och triangelns vinkelsumma''.}.
Detta eftersom att dessa fundamentala begrepp �r det som bygger upp
matematiken.
D�refter kan grundl�ggande m�ngdl�ra g�s igenom, med axiom och grundl�ggande
satser, och konstruktionen av de naturliga talen kan visas f�r att visa hur
matematiken fundamentalt �r uppbyggd.
F�r detaljer se Tabell \ref{tbl:inledning}.

\begin{table}[h]
	\caption{Planering f�r inledningen.}
	\begin{tabular}{ll}
		Inneh�ll & Tids�tg�ng (timmar) \\
		\hline\hline
		Logik, definition, sats och bevis. & 2 \\
		\hline
		M�ngder. & 1 \\
		\hline
		M�ngden av naturliga tal. & 1 \\
		\hline\hline
		& 4 \\
		\hline\hline\\
	\end{tabular}
	\label{tbl:inledning}
\end{table}

D�refter f�ljer l�mpligen avsnittet \emph{Taluppfattning, aritmetik
och algebra} eftersom att taluppfattning, aritmetik och algebra �r n�dv�ndiga
f�r vidare studier av matematiken, och allts� n�dv�ndiga f�r resten av kursen,
samt att det �r naturligt att studera heltalen efter de naturliga talen.
F�r detaljer se Tabell \ref{tbl:talteori}.
%Eftersom att avsnittet har fem punkter b�r \emph{25 timmar} avs�ttas f�r hela
%detta omr�de.

\begin{table}[h]
	\caption{Planering f�r avsnittet \emph{Taluppfattning, aritmetik och
		algebra}.}
	\begin{tabular}{ll}
		%\textbf{Inneh�ll} & \textbf{Tids�tg�ng (timmar)} \\
		Inneh�ll & Tids�tg�ng (timmar) \\
		\hline\hline
		Generalisering av aritmetikens r�kneregler till att\\
			hantera algebraiska uttryck. & 2 \\
		\hline
		Egenskaper hos m�ngden heltal, olika talbaser samt\\
			begreppen primtal och delbarhet. & 5 \\
		\hline
		Egenskaper hos m�ngderna rationella tal, irrationella\\
			tal och reella tal. & 5 \\
		\hline
		Begreppet linj�r olikhet. & 1 \\
		\hline
		Algebraiska och grafiska metoder f�r att l�sa linj�ra\\
			ekvationer, olikheter och potensekvationer. & 5 \\
		\hline
		Metoder f�r ber�kningar inom vardagslivet och\\
			karakt�rs�mnena med reella tal skrivna p� olika\\
			former, inklusive potenser med reella exponenter\\
			samt strategier f�r anv�ndning av digitala verktyg.
			& * \\
		\hline\hline
		& 21 \\
		\hline\hline\\
	\end{tabular}
	\label{tbl:talteori}
\end{table}

Efter egenskaperna hos heltal, rationella och irrationella tal och algebra �r
det kan funktionsbegreppet introduceras.
Funktionsbegreppet �r ett centralt begrepp inom all matematik och b�r g�s
igenom tidigt i kursen f�r att kunna anv�ndas senare i kursen, exempelvis n�r
de trigonometriska funktionerna och logaritmfunktionen introduceras.
F�r detaljer se Tabell \ref{tbl:funktioner}.

\begin{table}[h]
%	\caption{Planering f�r avsnittet \emph{Samband och f�r�ndring}, del 1.}
	\caption{Planering f�r avsnittet \emph{Samband och f�r�ndring}.}
	\begin{tabular}{ll}
		Inneh�ll & Tids�tg�ng (timmar) \\
		\hline\hline
		Begreppen funktion, definitions- och v�rdem�ngd samt\\
			egenskaper hos linj�ra funktioner samt potens-\\
			och exponentialfunktioner. Representationer av\\
			funktioner i form av ord, funktionsuttryck, tabeller\\
			och grafer.  & 6 \\
		\hline
		Skillnader mellan begreppen ekvation, olikhet, algebraiskt\\
			uttryck och funktion. & 2 \\
		\hline
%		\hline\hline
%		& 8 \\
%		\hline\\
%	\end{tabular}
%	\label{tbl:funktioner1}
%\end{table}
%\begin{table}
%	\caption{Planering f�r avsnittet \emph{Samband och f�r�ndring}, del 2.}
%	\begin{tabular}{ll}
%		Inneh�ll & Tids�tg�ng (timmar) \\
%		\hline\hline
		F�rdjupning av procentbegreppet: promille, ppm och\\
			procentenheter. & 2 \\
		\hline
		Begreppen f�r�ndringsfaktor och index samt metoder f�r\\
			ber�kning av r�ntor och amorteringar f�r olika typer\\
			av l�n. & 4 \\
		\hline\hline
%		& 6 \\
		& 14 \\
		\hline\hline\\
	\end{tabular}
	\label{tbl:funktioner}
\end{table}

D� f�reg�ende avsnitt \emph{Samband och f�r�ndring} avslutas med
procentbegreppet �r det l�mpligt att forts�tta med detta i avsnittet
\emph{Sannolikhet och statistik}.
Procentbegreppet anv�nds d� i en annan form �n ett m�tt av f�r�ndring och
eleverna f�r bygga vidare sin f�rst�else f�r hur procentbegreppet kan anv�ndas.
Sannolikhetsavsnittet �r ett relativt till kursen litet avsnitt, f�r detaljer
se Tabell \ref{tbl:sannolikhet}.

\begin{table}[h]
	\caption{Planering f�r avsnittet \emph{Sannolikhet och statistik}.}
	\begin{tabular}{ll}
		Inneh�ll & Tids�tg�ng (timmar) \\
		\hline\hline
		Begreppen beroende och oberoende h�ndelser samt\\
			metoder f�r ber�kning av sannolikheter vid slumpf�rs�k\\
			i flera steg med exempel fr�n spel och risk- och\\
			s�kerhetsbed�mningar. & 3 \\
		\hline
		Granskning av hur statistiska metoder och resultat anv�nds\\
			i samh�llet och i vetenskap. & 3 \\
		\hline\hline
		& 6 \\
		\hline\hline\\
	\end{tabular}
	\label{tbl:sannolikhet}
\end{table}

Det sista avsnittet i kursen �r avsnittet \emph{Geometri}.
Detta avsnitt �r ett avskiljt omr�de som har stor historisk betydelse f�r
matematiken.
I denna presentation kommer funktionsbegreppet att anv�ndas vid introduktionen
av de trigonometriska funktionerna, och d�rf�r b�r geometriavsnittet f�lja
efter introduktionen av funktionsbegreppet.
F�r detaljer se Tabell \ref{tbl:geometri}.

\begin{table}[h]
	\caption{Planering f�r avsnittet \emph{Geometri}.}
	\begin{tabular}{ll}
		Inneh�ll & Tids�tg�ng (timmar) \\
		\hline\hline
		Pythagoras sats & 1 \\
		\hline
		Begreppen sinus, cosinus och tangens samt metoder f�r\\
			ber�kning av vinklar och l�ngder i r�tvinkliga\\
			trianglar. & 4 \\
		\hline
		Begreppet vektor och dess representationer s�som riktad\\
			str�cka och punkt i ett koordinatsystem. & 1 \\
		\hline
		Addition och subtraktion med vektorer, produkten av en\\
			skal�r och en vektor samt skal�rprodukt av vektorer\\
			i ett koordinatsystem. & 3 \\
		\hline
		Pythagoras sats i \(N\) dimensioner. & 1 \\
		\hline\hline
		& 10 \\
		\hline\hline\\
	\end{tabular}
	\label{tbl:geometri}
\end{table}

Avslutningsvis sammanfattas kursplaneringen �versiktligt i Tabell
\ref{tbl:sammanfattning}.
Totalt �r 55 timmar avsatt f�r genomg�ngar, de �terst�ende 45 timmarna �r
oplanerade och b�r anv�ndas till exempelvis elevers redovisning av uppgifter,
klassrumsdiskussioner etc.
Dessa timmar skulle kunna spridas j�mnt �ver alla omr�den, men de b�r hellre
nyttjas d�r eleverna har intresse eller behov av mer tid.

\begin{table}[h]
	\caption{Kursplanerings�versikt.}
	\begin{tabular}{ll}
		Inneh�ll & Tids�tg�ng (timmar) \\
		\hline\hline
		Inledning & 4 \\
		Taluppfattning, aritmetik och algebra & 21 \\
		Samband och f�r�ndring & 14 \\
		Sannolikhet och statistik & 6 \\
		Geometri & 10 \\
		\hline\hline
		& 55 \\
		\hline
		Tid �ver & 45 \\
		\hline\hline\\
	\end{tabular}
	\label{tbl:sammanfattning}
\end{table}


\clearpage
\end{document}
