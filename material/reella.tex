\chapter{De reella talen}
\noindent
...

\begin{definition}[Algebraiska egenskaper hos de reella talen]
	P� m�ngden \(\R\) definieras tv� bin�ra operatorer, addition (\(+\)) och
	multiplikation (\(\cdot\)).
	F�r addition g�ller f�ljande:
	\begin{description}
		\item[Kommutativitet] \(a+b=b+a\) f�r alla \(a,b\in\R\).
		\item[Associtivitet] \((a+b)+c=a+(b+c)\) f�r alla \(a,b,c\in\R\).
		\item[Additiv enhet] Det finns ett element \(0\in\R\) s�dant att
			f�r alla \(a\in\R\) g�ller att \(0+a = a+0 = a\).
		\item[Additiv invers] F�r alla \(a\in\R\) finns ett element \(-a\in\R\)
			s�dant att \(a + (-a) = (-a) + a = 0\).
	\end{description}
	F�r multiplikation g�ller f�ljande:
	\begin{description}
		\item[Kommutativitet] \(a \cdot b=b \cdot a\) f�r alla \(a,b\in\R\).
		\item[Associtivitet] \((a \cdot b) \cdot c=a \cdot (b \cdot c)\) f�r
			alla \(a,b,c\in\R\).
		\item[Multiplikativ enhet] Det finns ett element \(1\in\R\) s�dant att
			f�r alla \(a\in\R\) g�ller att \(1 \cdot a = a \cdot 1 = a\).
		\item[Multiplikativ invers] F�r alla \(0 \neq a\in\R\) finns ett
			element \(1/a\in\R\) s�dant att
			\(a \cdot (1/a) = (1/a) \cdot a = 1\).
	\end{description}
	Ut�ver detta g�ller �ven
	\begin{description}
		\item[Multiplikativ distributet �ver addition]
			\(a \cdot (b+c) = (a \cdot b) + (a \cdot c)\) och
			\((b+c) \cdot a = (b \cdot a) + (c \cdot a)\) f�r alla reella tal
			\(a,b,c\in\R\).
	\end{description}
\end{definition}

\begin{exercise}
	Utforska vad de algebraiska egenskaperna hos de reella talen till�ter.
\end{exercise}
\begin{exercise}
	Om \(x\in\R\) och \(a\in\R\) b�da �r reella tal och \(x + a = a\),
	visa att \(x=0\).
\end{exercise}
\begin{exercise}
	Om \(x\in\R\) och \(a\in\R\) b�da �r reella tal och \(a \cdot x = a\),
	visa att \(x=1\).
\end{exercise}
\begin{exercise}
	Om \(a\in\R\) �r ett reellt tal, visa att \(a \cdot 0 = 0\).
\end{exercise}
\begin{exercise}
	Om \(0 \neq a \in \R\) och \(b\in\R\) �r reella tal och \(a \cdot b = 1\),
	visa att \(b = 1/a\).
\end{exercise}
\begin{exercise}
	Om \(a\in\R\) och \(b\in\R\) �r reella tal och \(a \cdot b = 0\),
	visa att antingen \(a=0\) eller \(b=0\), eller b�da.
\end{exercise}
