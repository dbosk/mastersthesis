\chapter{Introduktion}
\label{ch:Intro}
\nocite{Kiselman2008mfs}
% TODO skriv introduktionen.
\noindent
\lettrine{M}{atematiken}
har funnits i mer �n 5000 �r och �r m�nniskans l�ngst studerade �mne.

...



%%%%%%%%%%%%%%%%%%%%%%%%%%%%%%%%%%%%%
% VAD �R MATEMATIK?
%%%%%%%%%%%%%%%%%%%%%%%%%%%%%%%%%%%%%
\section{Vad �r matematik?}                                                     
% TODO skriv 'vad �r matematik'
% - Abstrakt konstruktion som endast finns i sinnet.
% - Naturvetenskap i sverige, filosofi i resten av v�rlden.
%   - Sveriges st�rta bidrag till matematikhistorien �r att vi tog d�d p� Ren�
%     Descartes (Touraine, 1596 -- Stockholm, 1650).
% - Beh�ver inte ha n�gon till synes uppenbar till�mpning, det �r vackert.
%   - Pierre de Fermat (160{1,7,8} -- 1665), advokat, upphovsman till Fermats
%     lilla sats och Fermats stora (sista) sats.
%   - Fermat-Euler satsen �r grunden f�r kryptosystemet n�r man loggar in p�
%     banken via Internet.
% - Demokratiskt organiserad. Alla kan bidra, ex. Fermat var advokat.
\noindent
...




\begin{table}
	\caption{Det grekiska alfabetet.}
	\begin{tabular}{lll}
		\(A\) & \(\alpha\) & alfa \\
		\(B\) & \(\beta\) & beta \\
		\(\Gamma\) & \(\gamma\) & gamma \\
		\(\Delta\) & \(\delta\) & delta \\
		\(E\) & \(\epsilon\) & epsilon \\
		\(Z\) & \(\zeta\) & zeta \\
		\(H\) & \(\eta\) & eta \\
		\(\Theta\) & \(\theta\) & theta \\
		\(I\) & \(\iota\) & iota \\
		\(K\) & \(\kappa\) & kappa \\
		\(\Lambda\) & \(\lambda\) & lambda \\
		\(M\) & \(\mu\) & my \\
	\end{tabular}
	\begin{tabular}{lll}
		\(N\) & \(\nu\) & ny \\
		\(\Xi\) & \(\xi\) & xi \\
		\(O\) & \(o\) & omikron \\
		\(\Pi\) & \(\pi\) & pi \\
		\(P\) & \(\rho\) & rho \\
		\(\Sigma\) & \(\sigma\) & sigma \\
		\(T\) & \(\tau\) & tau \\
		\(Y\) & \(\upsilon\) & ypsilon \\
		\(\Phi\) & \(\phi\) & phi \\
		\(X\) & \(\chi\) & khi \\
		\(\Psi\) & \(\psi\) & psi \\
		\(\Omega\) & \(\omega\) & omega \\
	\end{tabular}
	\label{tbl:greekalpha}
\end{table}
