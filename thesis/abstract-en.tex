\noindent
This study examines how formal mathematics can be taught in the Swedish 
secondary school with its new curriculum for mathematics. 
The study examines what a teaching material in formal mathematics 
corresponding to the initial content of the course Mathematics 1c could look
like, and whether formal mathematics can be taught to high school students. 

The survey was conducted with second year students from the science programme.
The majority of these students studied the course Mathematics D. 
The students described themselves as not being motivated towards mathematics.

The results show that the content of the curriculum can be presented with
formal mathematics. 
This both in terms of requirements for content and students being able to
comprehend this content.
The curriculum also requires that this type of mathematics is introduced in
the course Mathematics 1c.

The results also show that students are open towards and want more formal
mathematics in their ordinary education. 
They initially felt it was strange because they had never encountered this 
type of mathematics before, but
some students found the formal mathematics to be easier than the
mathematics ordinarily presented in class.

The study finds no reason to postpone the meeting with the formal mathematics 
to university level. 
Students' commitment to proof and their comprehention of content suggests that
formal mathematics can be introduced in high school courses. 
This study thus concludes that the new secondary school course Mathematics 1c
can be formalised and therefore makes possible a renewed mathematics education.
