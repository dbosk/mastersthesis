\noindent
This study examines how formal mathematics can be taught in the new Swedish 
secondary school with its new curriculum for mathematics. 
The study examines how a teaching material in formal mathematics 
corresponding to the initial content of the course Mathematics 1c could be, and 
whether that mathematics can be taught with high school students. 

The survey was conducted with second year students from the science programme.
These students studied the course Mathematics D. 
The students described themselves as non-mathematics motivated students. 

The results show that the content of the curriculum can be presented with
formal mathematics. 
This both in terms of time and requirements for content.
The curriculum also requires that this type of mathematics is introduced in
the course Mathematics 1c.

The results also show that students are open to formal mathematics. 
They initially felt it was strange because they have never encountered this 
type of mathematics before, but they still wanted more of it and they thought
it was missing in their own books.

The study finds no reason to postpone the meeting with the formal mathematics 
to university level. 
Students' commitment to proof and their comprehention of content suggests that
formal mathematics can be introduced in high school courses. 

This study concludes that the new secondary school course Mathematics 1c
apparently can be formalised.
But it also says that it would still require a longitudinal study to draw any
stronger conclusions.
