\noindent
Denna studie unders�ker hur formell matematik kan undervisas i den nya svenska
gymnasieskolan med dess nya �mnesplan f�r matematik.
I studien unders�ks hur ett undervisningsmaterial i formell matematik
motsvarande det inledande inneh�llet i kursen Matematik 1c kan se ut och
huruvida denna matematik kan undervisas med gymnasieelever.

Unders�kningen genomf�rdes med elever fr�n det naturvetenskapliga programmets
andra �rskurs.
Majoriteten av dessa elever l�ste d� kursen Matematik D.
Eleverna beskrev sig sj�lva som ej motiverade i matematik.

Resultatet visar att inneh�llet i �mnesplanen kan presenteras med formell
matematik.
Detta b�de med avseende �mnesplanens krav p� inneh�ll och att eleverna kan
f�rst� inneh�llet.
�mnesplanen kr�ver dessutom att denna typ av matematik tas upp som en del av
inneh�llet i kursen Matematik 1c.

Resultatet visar ocks� att eleverna �r �ppna f�r och vill ha mer formell
matematik i undervisningen.
De tyckte att det k�ndes ovant eftersom att de aldrig tidigare st�tt p� denna
typ av matematik, men vissa elever fann formell matematik som enklare �n
matematiken som normalt presenteras p� lektionerna.

Studien finner ingen anledning till att skjuta upp m�tet med formell matematik
till universitetsniv�.
Elevernas engagemang f�r bevis och tillgodog�randet av inneh�llet talar ocks�
f�r att formell matematik kan introduceras i gymnasiekurserna.
Studiens slutsats �r s�ledes att nya gymnasieskolans kurs Matematik 1c
kan formaliseras och �ppna upp f�r en f�rnyad matematikundervisning.
