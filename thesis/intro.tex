\chapter{Inledning}
\noindent
Dagens matematikundervisning har f�tt mycket kritik p� sistone, bland annat
fr�n Skolinspektionen.
Matematiken i undervisningen upplevs som r�knecentrerad [ref].
Eleverna l�r sig genom memorering snarare �n f�rst�else [ref].

B�de Skolinspektionens rapport om matematikundervisningen i grundskolan [ref],
PISA-unders�kningarna [ref] och TIMSS-rapporterna [ref] har visat detta.

...


\section{Syfte och fr�gest�llning}
\noindent
Detta projekts huvudsyfte �r att l�gga grunden f�r en formalisering av
gymnasiematematiken, genom att ta fram en b�rjan till undervisningsmaterial
motsvarande �mnesplanen f�r Matematik 1c.
Fokus kommer att ligga p� att ta fram ett kompendium med en 
inledning till rigor�s
matematik, en grund att bygga fortsatt undervisning p�.
Med rigor�s matematik avses matematik v�l grundad i logiskt resonemang med
satser bevisade utifr�n givna axiom.
Syftet �r s�ledes inte att ta fram ett f�rdjupningsmaterial, utan att
unders�ka om och hur dagens gymnasiematematik kan formaliseras.

St�d f�r projektet finns i den nya �mnesplanen Matematik 1c,
som f�reskriver h�gre krav p� rigor�st matematiskt inneh�ll.
Den nya �mnesplanen i matematik �r ett steg fram�t f�r det matematiska
inneh�llet i matematikundervisningen i svensk gymnasieskola.

