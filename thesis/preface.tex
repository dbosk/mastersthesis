\chapter{F�rord}
\noindent
Motivationen till denna studie har varit min k�rlek f�r matematik och dess
underbara generaliserbarhet samt min pedagogiska �vertygelse om att alla kan
l�ra sig matematik.
Jag har ocks� haft tankar om att den generaliserade matematiken med bevis och
h�rledningar ger mer f�rst�else till eleverna �n dagens undervisning i r�kning
med den evinnerliga repetitionen genom r�knande av tal efter tal.
Jag har m�tt m�nga som sett matematiken som en kombination av memorering av hur
siffror ska kombineras i olika situationer och i annat fall gissande av hur de
ska kombineras.
Undervisningen som den ser ut idag �r utformad att eleverna ska utifr�n ett
f�tal givna exempel f�rs�ka att generalisera kunskaper som de sedan ska anpassa
efter ett st�rre antal testuppgifter.
Jag har ocks�, inom det datalogiska omr�det maskininl�rning, sett algoritmer
som arbetar p� samma s�tt.
�r det d� m�jligt att s�ga att v�ra svenska matematikelever uppn�r en h�gre
f�rst�else f�r inneh�llet �n dessa datorprogram?
Jag har sj�lv sv�rt att finna en s�dan anledning.

Projektet har tagit mycket tid och kraft, men har varit otroligt roligt och
givande.
Jag skulle vilja tacka mina handledare Roy Skjelnes och Lil Engstr�m f�r
givande diskussioner kring arbetet.
Jag vill tacka Dan Laksov som ocks� deltagit under m�tena och bidragit med
v�rdefulla insikter.
Slutligen vill jag tacka den skola, den l�rare och de elever som deltagit i
unders�kningen.

\vspace{2cm}
\begin{flushright}
	\parbox{5cm}{
	S�r�ker, juni 2011\\
	Daniel Bosk
	}
\end{flushright}
